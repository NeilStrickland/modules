%%Autogenerated from modules.tex: do not edit


\BeginDeferredSolution{ex-which-ring}{2.1}
 \begin{itemize}
  \item $R_0$ is a ring.  The main point is to observe that $R_0$ is
   closed under addition and multiplication, because if $f(-x)=f(x)$
   and $g(-x)=g(x)$ then
   \begin{align*}
    (f+g)(-x) &= f(-x) + g(-x) = f(x) + g(x) = (f+g)(x) \\
    fg(-x) &= f(-x) g(-x) = f(x) g(x) = fg(x).
   \end{align*}
  \item $R_1$ is not a ring, because it is not closed under
   multiplication: if $f$ and $g$ lie in $R_1$ then
   \[ fg(-x)=f(-x)g(-x)=(-f(x))(-g(x))=+fg(x)\neq -fg(x), \]
   so $fg\not\in R_1$ (except in trivial cases where $fg=0$).
  \item $R_2$ is not a commutative ring, because matrix multiplication
   is not commutative in general.  For example, if we take
   $a=\bsm 1 & 0 \\ 0 & 0 \esm$ and $b=\bsm 0 & 1 \\ 0 & 0 \esm$ then
   $ab=b$ and $ba=0$ so $ab\neq ba$.  All the other axioms are
   satisfied, however.
  \item $R_3$ is a ring.  The additive identity is the zero matrix,
   and the multiplicative identity is the matrix $\bsm 1&1\\1&1\esm$.
  \item $R_4$ is not a ring.  Firstly, it is not commutative, because
   $b\tm a=-a\tm b$.  It is not even associative, because
   \begin{align*}
    a\tm (b\tm c) &= (a.c) b - (a.b) c \\
    (a\tm b)\tm c &= (a.c) b - (b.c) a.
   \end{align*}
   There is also no multiplicative identity: if there were, then we
   would have $1\tm 1=1$, but $a\tm a$ is always zero for any vector
   $a$.  ($R_3$ is in fact an example of a \emph{Lie algebra}; these
   are rather different from rings, but also very important.)
 \end{itemize}
\EndDeferredSolution

\BeginDeferredSolution{ex-typical-elts}{2.2}
 \begin{itemize}
 \item[(a)] In $\Z_{(5)}$ we could take $a=3/4$ and $b=6/7$; these
  both lie in $\Z_{(5)}$ because $4$ and $7$ are not divisible by $5$.
  We have $a+b=45/28$ and $ab=18/28=9/14$.  These both lie in
  $\Z_{(5)}$ because $28$ and $14$ are not divisible by $5$.
 \item[(b)] In $\Z[i]$ we could take $a=2+3i$ and $b=4-5i$.  We then
  have $a+b=6-2i$ and $ab=23+2i$; both of these clearly also lie in
  $\Z[i]$.
 \item[(c)] In $\Q[x,y]$ we could take $a=(x+y)/2$ and $b=(x-y)/2$.
  Then $a+b=x$ and $ab=(x^2-y^2)/4$, so $a+b$ and $ab$ are again
  elements of $\Q[x,y]$.
 \item[(d)] In $\Z_{12}$ we could take $a=\ov{3}$ and $b=\ov{4}$, so
  $a+b=\ov{7}=\ov{-5}$ and $ab=\ov{12}=\ov{0}$.
 \end{itemize}
\EndDeferredSolution

\BeginDeferredSolution{ex-six-local}{2.3}
 Put $a=1/2$ and $b=-1/3$.  Then $a,b\in R$ but $a+b=1/6\not\in R$ and
 $ab=-1/6\not\in R$, so $R$ is not closed under addition or
 multiplication.
\EndDeferredSolution

\BeginDeferredSolution{ex-field-domain}{2.4}
 Let $a$ and $b$ be nonzero elements of $K$; we must prove that
 $ab\neq 0$.  As $K$ is a field, we know that $a$ and $b$ are
 invertible, so we can find elements $c,d\in K$ with $ac=1$ and
 $bd=1$.  It follows that $abcd=1$.  If $ab$ were zero we would also
 have $abcd=0$, so $1$ would be equal to $0$, contradicting the
 definition of a field.  We must thus have $ab\neq 0$ as required.
\EndDeferredSolution

\BeginDeferredSolution{ex-PX-ring}{2.5}
 \begin{itemize}
  \item[(a)] We have $A\cup\emptyset=A$ and $A\cap\emptyset=\emptyset$
   so $A+\emptyset=A\sm\emptyset=A$.  Similarly, we have
   $A\cup A=A=A\cap A$, so $A+A=A\sm A=\emptyset$.
  \item[(b)] For the equation $\chi_{A+B}(x)=\chi_A(x)+\chi_B(x)$,
   there are four cases to consider.
   \begin{itemize}
   \item[(i)] $x$ lies in both $A$ and $B$, so $x$ does not lie in
    $A+B$, so $\chi_{A+B}(x)=\ov{0}$.  Here we have
    $\chi_A(x)+\chi_B(x)=\ov{1}+\ov{1}=\ov{2}=\ov{0}$ (because we are
    working in $\Z_2$), so $\chi_{A+B}(x)=\chi_A(x)+\chi_B(x)$ as
    required.
   \item[(ii)] $x$ lies in $A$ but not in $B$, so $x\in A+B$, so
    $\chi_{A+B}(x)=\ov{1}$.    Here we have
    $\chi_A(x)+\chi_B(x)=\ov{1}+\ov{0}=\ov{1}=\chi_{A+B}(x)$ as
    required.
   \item[(iii)] $x$ lies in $B$ but not in $A$; this works the same
    way as in~(ii).
   \item[(iv)] $x$ lies in neither $A$ nor $B$.  Here it is clear that
    $\chi_{A+B}(x)=\ov{0}=\ov{0}+\ov{0}=\chi_A(x)+\chi_B(x)$.
   \end{itemize}
   The argument is similar but easier for the equation
   $\chi_{AB}(x)=\chi_A(x)\chi_B(x)$.
  \item[(c)] It is clear that $\{x\in X\st \chi_A(x)=\ov{1}\}=A$.  If
   $\chi_A=\chi_B$ then
   $\{x\st\chi_A(x)=\ov{1}\}=\{x\st\chi_B(x)=\ov{1}\}$, so $A=B$.
  \item[(d)]
   It is clear that the above rules do indeed define subsets of $X$, so
   $R$ is closed under addition and multiplication.  It is easy to see
   that $AB=A\cap B=B\cap A=BA$ and
   $(AB)C=(A\cap B)\cap C=A\cap(B\cap C)=A(BC)$, so multiplication is
   commutative and associative.  Moreover, for $A\sse X$ we have
   $X\cap A=A$, so $X$ is a multiplicative identity element.

   It is also clear that $A+B=B+A$, so addition and is commutative.
   Part~(a) says that $\emptyset$ is an additive identity, and $A$ is
   an additive inverse for itself.  We next show that addition is
   associative.  By part~(b), we have
   \[ \chi_{A+(B+C)}(x) = \chi_A(x)+\chi_{B+C}(x)
       = \chi_A(x) + \chi_B(x) + \chi_C(x) =
       \chi_{A+B}(x) + \chi_C(x) = \chi_{(A+B)+C}(x).
   \]
   It follows using~(c) that $A+(B+C)=(A+B)+C$, as required.

   All that is left is to check distributivity, which can be done by
   the same method.  We have
   \begin{align*}
    \chi_{A(B+C)}(x) &= \chi_A(x) \chi_{B+C}(x) \\
                     &= \chi_A(x) (\chi_B(x) + \chi_C(x)) \\
                     &= \chi_A(x)\chi_B(x) + \chi_A(x)\chi_C(x) \\
                     &= \chi_{AB}(x) + \chi_{AC}(x) \\
                     &= \chi_{AB+AC}(x),
   \end{align*}
   so $A(B+C)=AB+AC$ as required.
 \end{itemize}
\EndDeferredSolution

\BeginDeferredSolution{ex-Z-ten}{3.1}
 The elements are
 $(\ov{0},\ov{0}),(\ov{0},\ov{1}),(\ov{0},\ov{2}),(\ov{0},\ov{3}),
 (\ov{0},\ov{4}),(\ov{1},\ov{0}),(\ov{1},\ov{1}),(\ov{1},\ov{2}),
 (\ov{1},\ov{3})$ and $(\ov{1},\ov{4})$.  I claim that the element
 $x:=(\ov{1},\ov{1})$ has order $10$.  Indeed, we have
 $nx=(\ov{n},\ov{n})$.  The first $\ov{n}$ is in $\Z_2$, so it is zero
 iff $n$ is divisible by $2$.  The second $\ov{n}$ is in $\Z_5$, so it
 is zero iff $n$ is divisible by $5$.  Thus $nx=(\ov{0},\ov{0})$ iff
 $n$ is divisible by both $2$ and $5$, or equivalently iff $n$ is
 divisible by $10$.  This means that $x$ has order $10$ as claimed.
\EndDeferredSolution

\BeginDeferredSolution{ex-diffop-calc}{3.2}
 \begin{itemize}
 \item[(a)] We have $D.t^3=3t^2$ so $D^2.t^3=3D.t^2=6t$ so
  $D^3.t^3=6D.t=6$.  This means that $(D^2/2).t^3=3t$ and
  $(D^3/6).t^3=1$ so $(1+D+D^2/2+D^3/6).t^3=t^3+3t^2+3t+1$.  We notice
  that this is just $(t+1)^3$.  The generalisation is that
  \[ (\sum_{k=0}^m D^k/k!).t^m = (t+1)^m. \]
  More generally, if $f(t)$ is any polynomial of degree less than or
  equal to $m$, it can be shown that
  \[ (\sum_{k=0}^m D^k/k!).f(t) = f(t+1). \]
  This is essentially Taylor's theorem.
 \item[(b)] Put
  \[ g(t)=((D+1)f)(t)=f'(t)+f(t)=
       (-e^{-t}\sin(t)+e^{-t}\cos(t))+e^{-t}\sin(t)=e^{-t}\cos(t).
  \]
  Then $((D+1)^2f)(t)=((D+1)g)(t)=g'(t)+g(t)=-e^{-t}\sin(t)$, by a
  similar calculation.  In other words $(D+1)^2f=-f$.
 \item[(c)] We have
  $((D-1)g_k)(t)=g'_k(t)-g_k(t)=(kt^{k-1}e^t+t^ke^t)-t^ke^t=kt^{k-1}e^t$,
  or in other words $(D-1)g_k=kg_{k-1}$.  It follows that
  \begin{align*}
   (D-1)^2g_k &= k(D-1)g_{k-1}=k(k-1)g_{k-2} \\
   (D-1)^3g_k &= k(k-1)(D-1)g_{k-2}=k(k-1)(k-2)g_{k-3}
  \end{align*}
  and so on.  We eventually find that $(D-1)^kg_k=k!g_0$, so
  $((D-1)^kg_k)(t)=k!e^t$.
 \item[(d)] We certainly have $(D^0f)(t)=f(t)=(0+t)e^t$.  Assuming
  that $(D^kf)(t)=(k+t)e^t$ for some particular value of $k$, we have
  \[ (D^{k+1}f)(t) = D((k+t)e^t)= (k+t) D(e^t) + e^t D(k+t) =
      (k+t) e^t + e^t = ((k+1) + t) e^t.
  \]
  It follows by induction that $(D^kf)(t)=(k+t)e^t$ for all $k$.
  Thus, for an operator $p(D)=\sum_k a_kD^k$, we have
  \[ (p(D)f)(t) = \sum_k a_k(k+t) e^t =
       ((\sum_k ka_k) + (\sum_k a_k)t) e^t.
  \]
  We also have $p(1)=\sum_k a_k.1^k=\sum_ka_k$.  Similarly, we have
  $p'(D)=\sum_k ka_k D^{k-1}$, so $p'(1)=\sum_k ka_k$.  Putting these
  into our earlier formula gives
  \[ (p(D)f)(t) = (p'(1) + p(1)t) e^t, \]
  as claimed.
 \end{itemize}
\EndDeferredSolution

\BeginDeferredSolution{ex-gaussian-poly}{3.3}
 \begin{itemize}
  \item[(a)] We just need to check that $V$ is closed under
   differentiation.  Note that $v'(t)=te^{t^2/2}=tv(t)$, so
   \[ \frac{d}{dt}f(t)v(t) = f(t)v'(t) + f'(t)v(t) =
        (tf(t)+f'(t)) v(t).
   \]
   If $f(t)$ is a polynomial, then clearly $tf(t)+f'(t)$ is also a
   polynomial, so the function $(tf(t)+f'(t))v(t)$ lies in $V$ as
   required.
  \item[(b)] If we differentiate repeatedly using the above rule we
   find that
   \begin{align*}
    (D^0 v)(t) &=   v(t) \\
    (D^1 v)(t) &= t v(t) \\
    (D^2 v)(t) &= (t^2+1) v(t) \\
    (D^3 v)(t) &= (t^3+3t) v(t).
   \end{align*}
  \item[(c)] As $V$ is an $\R[D]$-module, we must have $D^kv\in V$, so
   $D^kv=p_kv$ for some polynomial $p_k$.  (From part~(b) we see that
   $p_0(t)=1$, $p_1(t)=t$, $p_2(t)=t^2+1$ and $p_3(t)=t^3+3t$.)  Using
   part~(a) we see that $p_{k+1}(t)=tp_k(t)+p'_k(t)$.  The
   claim is that $p_k(t)=t^k+\text{ lower terms }$.  If this is true
   for some value of $k$, then $tp_k(t)=t^{k+1}+\text{ lower terms }$
   and $p'_k(t)=kt^{k-1}+\text{ lower terms }$, so
   $p_{k+1}(t)=t^{k+1}+\text{ lower terms }$, so the claim holds for
   the next value of $k$.  Moreover, the claim visibly holds for
   $k=0$, so it holds for all $k$ by induction.
  \item[(d)] Let $k$ be the degree of $q$, so
   $q(D)=a_0+a_1D+\ldots+a_kD^k$ for some $a_0,\ldots,a_k\in\R$ with
   $a_k\neq 0$.  Then $q(D)v=(a_0p_0+\ldots+a_kp_k)v$, and using
   part~(c) we see that
   $a_0p_0(t)+\ldots+a_kp_k(t)=a_kt^k+\text{ lower terms }$, so in
   particular it is not zero.
  \item[(e)] First suppose that $f$ has degree $0$, say $f(t)=c$ for
   all $t$, where $c\in\R$.  We can regard $c$ as an element in
   $\R[D]$, and $fv=cv$ as required; this proves the claim for $k=0$.

   Now suppose we have proved the claim for all polynomials of degree
   less than $k$, and that $f$ has degree $k$.  Then
   $f(t)=at^k+\text{ lower terms }$ for some $a\in\R$.  It follows
   that the function $g(t)=f(t)-ap_k(t)$ is a polynomial of degree
   less than $k$, so we have $gv=q(D)v$ for some $q(D)\in\R[D]$.  It
   follows that
   \[ fv = gv + ap_kv = q(D)v + a D^kv = (q(D)+aD^k)v, \]
   so $fv\in\R[D]v$ as required.  The claim now follows for all
   degrees by induction.
  \item[(f)] We know from part~(e) that $v$ generates $V$ as an
   $\R[D]$-module.  It follows that $V\simeq\R[D]/I$, where
   $I=\{q(D)\in\R[D]\st q(D)v=0\}$.  Part~(d) tells us that $I=0$, so
   $V\simeq\R[D]$.
 \end{itemize}
\EndDeferredSolution

\BeginDeferredSolution{ex-jordan-powers}{4.1}
 The first few powers are
 \begin{align*}
  A^2 &= \bsm 1&2&1&0 \\
              0&1&2&1 \\
              0&0&1&2 \\
              0&0&0&1 \esm \\
  A^3 &= \bsm 1&3&3&1 \\
              0&1&3&3 \\
              0&0&1&3 \\
              0&0&0&1 \esm \\
  A^4 &= \bsm 1&4&6&4 \\
              0&1&4&6 \\
              0&0&1&4 \\
              0&0&0&1 \esm
 \end{align*}
 In general, in the matrix $A^n$ all the entries in the $k$'th band
 above and parallel to the diagonal are equal to the binomial
 coefficient $\bsm n\\k\esm$.  The entries below the diagonal are
 zero.
\EndDeferredSolution

\BeginDeferredSolution{ex-MA-calc}{4.2}
 \begin{itemize}
  \item[(a)] $A^3-I=\bsm 0&0&3\\0&0&0\\0&0&0\esm$ so
   \[ (x^3-1)m = \bsm 0&0&3\\0&0&0\\0&0&0\esm\bsm 1\\1\\1\esm
               = \bsm 3\\0\\0\esm.
   \]
  \item[(b)] $A^2=I$ so $A^3=A$ so $A^3-I=A-I$.  Moreover $Am=(3,2,1)$
   so $(A-I)m=Am-m=(3,2,1)-(1,2,3)=(2,0,-2)$.  Thus
   $(x^3-1)m=(2,0,-2)$.
  \item[(c)] $Am=(0,0)$ so $x^km=A^km=0$ for all $k>0$.  Thus when we
   expand out $(14x^{12}+5x^{11}-36x^7-22x^4+13x-5)m$, all the terms
   except the last one are zero, so we are left with $-5m=(-5,5)$.
 \end{itemize}
\EndDeferredSolution

\BeginDeferredSolution{ex-cayley-two}{4.3}
 We have
 \[ A^2 = \bbm a&b\\c&d\ebm\bbm a&b\\c&d\ebm
        = \bbm a^2 + bc & ab + bd \\ ac + cd & bc+d^2 \ebm
 \]
 and
 \[ (a+d)A = \bbm a^2+ad & ab+bd \\ ac + cd & ad + d^2 \ebm \]
 so
 \begin{align*}
  f(A) &= A^2 - (a+d)A + (ad-bc)I \\
       &= \bbm a^2 + bc & ab + bd \\ ac + cd & bc+d^2 \ebm
          - \bbm a^2+ad & ab+bd \\ ac + cd & ad + d^2 \ebm +
          \bbm ad-bc & 0 \\ 0 & ad-bc \ebm \\
       &= \bbm 0 & 0\\0 & 0\ebm.
 \end{align*}
\EndDeferredSolution

\BeginDeferredSolution{ex-fA-block}{4.4}
 $A^2=\bsm 2&2\\2&2\esm=2A$ so $A^4=(A^2)^2=(2A)^2=4A^2=8A$.  Thus
 $f(A)=A^4-3A=8A-3A=5A=\bsm 5&5\\5&5\esm$.
\EndDeferredSolution

\BeginDeferredSolution{ex-fA-swap}{4.5}
 \begin{itemize}
  \item[(a)] Clearly $A^0=I$ and $A^1=A$.  We observe that $A^2=I$,
   and it follows immediately that $A^i$ is $I$ whenever $i$ is even,
   and $A$ whenever $i$ is odd.
  \item[(b)] We have $f(1)=\sum_ia_i=b+c$, and
   $f(-1)=\sum_i(-1)^ia_i=b-c$.  It follows that $b=(f(1)+f(-1))/2$
   and $c=(f(1)-f(-1))/2$.
  \item[(c)] We have $f(A)=\sum_ia_iA^i$.  The term corresponding to
   an even number $i=2j$ is $a_{2j}I$, whereas the term corresponding
   to an odd number $i=2j+1$ is $a_{2j+1}A$.  We thus have
   \begin{align*}
    f(A) &= \sum_j (a_{2j} I + a_{2j+1} A)  \\
         &= b I + c A \\
         &= ((f(1) + f(-1))/2) I + ((f(1) - f(-1))/2) A \\
         &= f(1) (I + A)/2 + f(-1) (I - A)/2 \\
         &= \frac{f(1)}{2}\bbm 1&1\\1&1\ebm +
            \frac{f(-1)}{2}\bbm 1&-1\\-1&1\ebm.
   \end{align*}
 \end{itemize}
\EndDeferredSolution

\BeginDeferredSolution{ex-image-annihilator}{5.1}
 Let $\al\:M\xra{}N$ be a homomorphism of modules over a Euclidean
 domain $R$.  Suppose that $aM=\{0\}$ and $bN=\{0\}$ and let $c$ be
 the gcd of $a$ and $b$.  I claim that $cn=0$ for all
 $n\in\img(\al)$.  Indeed, we can write $c=au+bv$ for some
 $a,b\in R$.  If $n\in\img(\al)$ then $n=\al(m)$ for some $m\in M$.
 We have $am=0$ (because $aM=\{0\}$) so
 $an=a\al(m)=\al(am)=\al(0)=0$.  We also have $n\in N$ and $bN=\{0\}$
 so $bn=0$.  Thus $cn=uan+vbn=0+0=0$ as claimed.

 In the case considered we have $R=\C[x]$ and $a=x^3-x=(x-1)(x+1)x$
 and $b=x^5$ so it is clear that $c=x$.  Thus $xn=0$ for all
 $n\in\img(\al)$.
\EndDeferredSolution

\BeginDeferredSolution{ex-cyclic-retract}{5.2}
 Suppose that $M\op N$ is cyclic, so there is an element
 $(x,y)\in M\op N$ as an $R$-module.  This means that for any element
 $(m,n)\in M\op N$, there exists $a\in R$ such that $a(x,y)=(m,n)$.
 In particular, for any element $m\in M$ we have $(m,0)\in M\op N$, so
 there exists $a\in R$ such that $a(x,y)=(m,0)$, which means that
 $m=ax$.  This shows that $M$ is generated by the single element $x$,
 so $M$ is cyclic.  Similarly, $N$ is generated by $y$ and so is
 cyclic.
\EndDeferredSolution

\BeginDeferredSolution{ex-which-submodule}{5.3}
 \begin{itemize}
  \item[(a)] The set $N_1$ is not a submodule, because $(1,0)\in N_1$
   but $2(1,0)=(2,0)\not\in N_1$.  However, the set $N_0$ is a
   submodule.  To see this, suppose that $(n,m)$ and $(n',m')$ lie in
   $N_0$, so $n-m$ and $n'-m'$ are even.  Then
   $(n,m)+(n',m')=(n+n',m+m')$ and the integer
   $(n+n')-(m+m')=(n-m)+(n'-m')$ is even so $(n,m)+(n',m')\in N_0$.
   Similarly, for any $a\in\Z$ we have $a(n,m)=(an,am)$ and
   $an-am=a(n-m)$ is even, so $a(n,m)\in N_0$.  It is clear that
   $(0,0)\in N_0$ and it follows that $N_0$ is a submodule as
   claimed.
  \item[(b)] I claim that both $N_0$ and $N_1$ are submodules of
   $M_A$.  It is clear that they are both vector subspaces of $\R^2$,
   so it is enough to check that they are both stable under $A$.  An
   element $w\in N_0$ has the form $w=\bsm x\\x\esm$ so
   $Aw=\bsm 1&1\\1&1\esm\bsm x\\x\esm=\bsm 2x\\2x\esm$, so
   $Aw\in N_0$.  This shows that $N_0$ is stable under $A$ and thus is
   a submodule.  Similarly, an element $w\in N_1$ has the form
   $w=\bsm x\\-x\esm$ so $Aw=\bsm 0\\0\esm$.  As the zero vector
   certainly lies in $N_1$ we have $Aw\in N_1$ and so $N_1$ is also
   stable under $A$.
  \item[(c)] I claim that neither $N_0$ nor $N_1$ is an
   $\R[D]$-submodule of $\CRR$.  Indeed, put $f(t)=t-1$, so
   $f\in\CRR$.  Then $f(0)=0$, so $f\in N_0$.  However,
   $f'(0)=1\neq 0$, so $f'\not\in N_0$, so $N_0$ is not closed under
   differentiation, so it is not an $\R[D]$-submodule of $\CRR$.
   To prove that $N_1$ is not a submodule, we can use the same
   function $f$.  We have $\int_0^2f=[t^2/2-t]_0^2=0$, so $f\in N_1$.
   However, $\int_0^2f'=\int_0^21=2$, so $f'\not\in N_1$, so $N_1$ is
   not a submodule.
 \end{itemize}
\EndDeferredSolution

\BeginDeferredSolution{ex-ten-elements}{5.4}
 If $M$ were free, it would be isomorphic to $R^d$ for some $d$, so we
 would have $20=|M|=|R^d|=|R|^d=10^d$.  As $20$ is not a power of
 $10$, this is impossible, so $M$ cannot be free.
\EndDeferredSolution

\BeginDeferredSolution{ex-twenty-four}{5.5}
 The elements of $N_6$ are the multiples of $\ov{6}$, which are
 $\ov{0}$, $\ov{6}$, $\ov{12}$ and $\ov{18}$.  We can stop at this
 point because $\ov{24}=\ov{0}$, $\ov{30}=\ov{6}$ and so on.  Thus
 $N_6=\{\ov{0},\ov{6},\ov{12},\ov{18}\}$, and similarly we have
 $N_4=\{\ov{0},\ov{4},\ov{8},\ov{12},\ov{16},\ov{20}\}$.  From this we
 see that $N_4\cap N_6=\{\ov{0},\ov{12}\}=N_{12}$, so we can take
 $d=12$.

 As $\ov{8}\in N_4$ and $\ov{18}\in N_6$, the group $N_4+N_6$ contains
 $\ov{8}+\ov{18}=\ov{26}=\ov{2}$.  As $N_4+N_6$ is a subgroup of
 $\Z_{24}$ we deduce that all multiples of $\ov{2}$ lie in $N_4+N_6$,
 so $N_2\sse N_4+N_6$.  On the other hand, as $4$, $6$ and $24$ are
 all even we see that all elements of $N_4+N_6$ have the form $\ov{a}$
 for some even integer $a$ and thus they lie in $N_2$.  This shows
 that $N_4+N_6=N_2$, so we can take $e=2$.
\EndDeferredSolution

\BeginDeferredSolution{ex-nine-hundred}{5.6}
 \begin{itemize}
  \item[(a)] The order is $900/10=90$.
  \item[(b)] The factor group $\Z_{900}/N_{10}$ is isomorphic to
   $\Z_{10}$.
  \item[(c)] Here $d$ is the greatest common divisor of $70$ and
   $900$, which is $10$.
  \item[(d)] Here $d$ is the greatest common divisor of $12=2^2\tm 3$,
   $30=2\tm 3\tm 5$ and $100=2^2\tm 5^2$, so $d=2$.
  \item[(e)] Here $d$ is the least common multiple of $30=2\tm 3\tm 5$
   and $50=2\tm 5^2$, so $d=2\tm 3\tm 5^2=150$.
 \end{itemize}
\EndDeferredSolution

\BeginDeferredSolution{ex-MA-cyclic}{5.7}
 Put $u=(1,0,0)$.  Then $xu=(0,0,1)$ and $x^2u=(1,1,1)$.  These three
 vectors are clearly linear independent, so they form a basis of
 $\Q^3$, so they span $\Q^3$.  Thus any vector $v\in\Q^3$ can be
 written in the form $au+bxu+cx^2u$ for some $a,b,c\in\Q$.  In other
 words, $v=(a+bx+cx^2)u\in\Q[x]u$, so we see that $u$ generates $M_A$
 as a $\Q[x]$-module.  This means that $M_A\simeq\Q[x]/f(x)$ for some
 polynomial $f(x)$, which we can assume is monic.  From the general
 theory we know that the degree of $f$ is the size of $A$, which is
 $3$.  We also know that $f(x)$ is the only monic polynomial of degree
 $3$ such that $f(x).u=0$.

 The polynomial $f$ is in fact the characteristic polynomial of $A$,
 which can be calculated directly:
 \[ \det(xI-A) = \left| \begin{array}{ccc}
     x & 0 & -1 \\ 0 & x-1 & -1 \\ -1 & -1 & x-1
    \end{array} \right| =
    x^3 - 2 x^2 - x + 1.
 \]

 For another approach, consider the vector $x^3u=x(1,1,1)=(1,2,3)$.
 We would like to write this in the form $au+bxu+cx^2u$, so we want
 \[ (1,2,3)=a(1,0,0) + b(0,0,1) + c(1,1,1)=(a+c,c,b+c). \]
 The solution is $a=-1,b=1,c=2$, so $x^3u=-u+xu+2x^2u$, so
 $(x^3-2x^2-x+1)u=0$, so $f(x)=x^3- 2 x^2 - x + 1$.
\EndDeferredSolution

\BeginDeferredSolution{ex-Wd-cyclic}{5.8}
 Define $\al\:\R[D]\xra{}W_d$ by $\al(p(D))=p(D).t^d$.  Note that
 \begin{align*}
  D  .t^d &= d t^{d-1} \\
  D^2.t^d &= d(d-1) t^{d-2} \\
  D^3.t^d &= d(d-1)(d-2) t^{d-3}
 \end{align*}
 and so on.  In general, we have $D^k.t^d=m_kt^{d-k}$, where
 $m_k=d(d-1)\ldots(d-k+1)=\prod_{i=0}^{k-1}(d-i)$, as one can easily
 check by induction.  Note also that when $k\leq d$ all the factors
 $d-i$ for $0\leq i\leq k-1$ are nonzero, so $m_k\neq 0$.  However, we
 have $m_k=0$ for $k>d$.  Any element $f(t)\in W_d$ has the form
 $f(t)=a_0+a_1t+\ldots+a_dt^d$ for some $a_0,\ldots,a_d\in\R$.  If we
 define
 \[ p(D)= \sum_{i=0}^d a_{d-i} D^i/m_i =
          a_d+a_{d-1}D/m_1+\ldots+a_0D^d/m_d
 \]
 we find that
 \begin{align*}
  p(D).t^d &=
   a_d t^d + a_{d-1} m_1^{-1} D.t^d + \ldots + a_0 m_d^{-1} D^d.t^d \\
  &= a_d t^d + a_{d-1} t^{d-1} + \ldots + a_0 t^0 \\
  &= f(t).
 \end{align*}
 Thus every element $f\in W_d$ has the form $f=p(D).t^d$ for some
 $p(D)\in\R[D]$, so $W_d$ is generated by $t^d$ as a module over
 $\R[D]$.

 Now put $I=\{p\in\R[D]\st p(D).t^d=0\}$, so $W_d\simeq\R[D]/I$.
 Suppose we have an element $p(D)=\sum_ib_iD^i\in\R[D]$.  Then
 $p(D).t^d=b_0t^d+b_1m_1t^{d-1}+\ldots+b_dm_dt^0$, and this is zero
 iff $b_0=\ldots=b_d=0$.  This means that $p(D)$ has the form
 $b_{d+1}D^{d+1}+b_{d+2}D^{d+2}+\ldots$, so it is divisible by
 $D^{d+1}$.  Thus $I$ is the principal ideal $\R[D].D^{d+1}$, and
 $W_d\simeq\R[D]/D^{d+1}$.
\EndDeferredSolution

\BeginDeferredSolution{ex-crossprod}{5.9}
 \begin{itemize}
 \item[(a)] If $v\in L$ then $v=tu$ for some $t\in\R$, so
  $xv=\phi(v)=t(u\tm u)=0$ (using the fact that $a\tm a=0$ for any
  vector $a$).  This shows that $xL=\phi(L)=0$, so certainly
  $\phi(L)\leq L$, so $L$ is a submodule.
 \item[(b)] For any $v\in M$ we have $xv=\phi(v)=u\tm v$, which is
  always perpendicular to $u$, so it lies in $K$.  This says that
  $xM=\phi(M)\leq K$, so certainly $\phi(K)\leq K$, so $K$ is a
  submodule.  Next, for any $v\in K$ we have $u.v=0$ and so
  \[ x^2v = \phi^2(v) = u\tm(u\tm v) = (u.v)u - (u.u)v =
      0u - r^2v = -r^2 v,
  \]
  so $(x^2 + r^2)v=0$.  This shows that $(x^2+r^2)K=0$ as claimed.
 \item[(c)] We now know that $xM\leq K$ so
  \[ (x^3+r^2x)M = (x^2+r^2)xM \leq (x^2 + r^2) K = 0. \]
 \end{itemize}
\EndDeferredSolution

\BeginDeferredSolution{ex-Hom-MA-MB-i}{6.1}
 Such homomorphisms correspond to matrices $P$ of the appropriate
 shape (the same number of columns as $A$, and the same number of rows
 as $B$) such that $PA=BP$.
 \begin{itemize}
  \item[(a)] Here $P$ is a $2\tm 2$ matrix, say $P=\bsm a&b\\c&d\esm$.
   We have $PA=\bsm a&a+b\\c&c+d\esm$ and $BP=\bsm c&d\\a&b\esm$.  Thus
   $PA=BP$ if and only if $a=c$, $a+b=d$, $c=a$ and $c+d=b$.  By
   solving these equations we find that $c=a=0$ and $d=b$.  Thus, the
   homomorphisms from $M_A$ to $M_B$ are precisely the matrices of the
   form $P=\bsm 0&b\\0&b\esm$ over $\Q$.
  \item[(b)] Here again we have $P=\bsm a&b\\c&d\esm$ for some
   $a,b,c,d$.  Thus $PA=\bsm \lm a&\mu b\\ \lm c&\mu d\esm$ and
   $BP=\bsm\mu a&\mu b\\ \lm c&\lm d\esm$, so $PA=BP$ if and only if
   $\lm a=\mu a$ and $\mu d=\lm d$, or in other words
   $(\lm-\mu)a=(\lm-\mu)d=0$.  As $\lm-\mu\neq 0$ this means that
   $a=d=0$.  Thus, the homomorphisms from $M_A$ to $M_B$ are precisely
   the matrices of the form $P=\bsm 0&b\\ c&0\esm$.
  \item[(c)] Here $P$ is a $2\tm 3$ matrix over $\Q$.  We have
   $PA=PI_2=P$ and $BP=I_3P=P$ so the condition $PA=BP$ is
   automatically satisfied.  Thus the homomorphisms from $M_A$ to
   $M_B$ are all the $2\tm 3$ matrices over $\Q$.
 \end{itemize}
\EndDeferredSolution

\BeginDeferredSolution{ex-Hom-MA-MB-ii}{6.2}
 \begin{itemize}
 \item[(a)] The homomorphisms correspond to matrices
  $C=\bsm a&b\\c&d\esm$ with $CA=BC$, or equivalently
  $\bsm a&2a+2b\\c&2c+2d\esm=\bsm a+c&b+d\\a+c&b+d\esm$, so
  \begin{align*}
   a &= a+c \\
   2a+2b &= b+d \\
   c &= a+c \\
   2c+2d &= b+d.
  \end{align*}
  The first and third equations give $a=c=0$, and the remaining
  equations then give $d=b$, so $C$ must have the form
  $\bsm 0&b\\0&b\esm$.
 \item[(b)] We need to find the $3\tm 3$ matrices with $CA=BC$.  Note
  that $BC=C$.  Put $D=A-I=\bsm 0&0&0\\1&0&0\\0&1&0\esm$.  The
  condition $CA=BC$ now becomes $C(I+D)=C$, or equivalently $CD=0$.
  If the columns of $C$ are $u$, $v$ and $w$, then the columns of $CD$
  are easily seen to be $v$, $w$ and $0$.  Thus $CD=0$ iff $v=w=0$, so
  all the nonzero entries of $C$ must be in the first column.  Thus
  the homomorphisms from $M_A$ to $M_B$ are the matrices of the form
  $\bsm a&0&0\\b&0&0\\c&0&0\esm$.
 \item[(c)] Here we need the matrices
  $C=\bbm c_{11}&c_{21}&c_{31}\\c_{12}&c_{22}&c_{32}\ebm$ such that
  \[ \bbm c_{11}&c_{21}&c_{31}\\c_{12}&c_{22}&c_{32}\ebm
     \bbm \lm_1&0&0\\0&\lm_2&0\\0&0&\lm_3\ebm =
     \bbm \mu_1&0 \\ 0&\mu_2 \ebm
     \bbm c_{11}&c_{21}&c_{31}\\c_{12}&c_{22}&c_{32}\ebm,
  \]
  or equivalently
  \[ \bbm (\lm_1-\mu_1)c_{11} &
          (\lm_2-\mu_1)c_{21} &
          (\lm_3-\mu_1)c_{31} \\
          (\lm_1-\mu_2)c_{11} &
          (\lm_2-\mu_2)c_{21} &
          (\lm_3-\mu_2)c_{31} \ebm =
     \bbm 0&0&0 \\ 0&0&0 \ebm.
  \]
  For most choices of numbers $\lm_1,\lm_2,\lm_3,\mu_1,\mu_2$, the
  $\lm$'s will all be different from the $\mu$'s, so all the numbers
  $\lm_i-\mu_j$ will be nonzero.  In this case, the only possible
  matrix $C$ is the zero matrix.  In general, if we have $\lm_i=\mu_j$
  for some pairs $(i,j)$, then the corresponding entries $c_{ij}$ can
  be nonzero.
 \end{itemize}
\EndDeferredSolution

\BeginDeferredSolution{ex-annihilator}{6.3}
 \begin{itemize}
  \item[(a)]
   $\ann(4,\Z_{12})=\{\ov{0},\ov{3},\ov{6},\ov{9}\}\simeq\Z_4$.
  \item[(b)]
   We have
   \[ (x-1)(u,v) = (A-I)(u,v) = \bbm 0&0\\1&0 \ebm\bbm u\\ v\ebm =
       \bbm 0 \\ u \ebm,
   \]
   so $(x-1)(u,v)=(0,0)$ iff $u=0$.  Thus
   $\ann(x-1,M_A)=\{(0,v)\st v\in\R\}$.
  \item[(c)] $a\al(\ov{1})=\al(a.\ov{1})=\al(\ov{a})=\al(\ov{0})=0$,
   because $\ov{a}=\ov{0}$ in $R/a$.
  \item[(d)] Suppose that $m\in\ann(a,M)$.  We can certainly define a
   homomorphism $\bt\:R\xra{}M$ by $\bt(x)=xm$.  As $am=0$ this satisfies
   $\bt(ya)=yam=0$, so $\bt(x)=0$ for $x\in Ra$.  By the first
   isomorphism theorem we get an induced map
   $\al=\ov{\bt}\:R/aR\xra{}M$ defined by $\al(\ov{x})=\bt(x)=xm$, and
   in particular $\al(\ov{1})=m$.
  \item[(e)] Note that $\ann(D^2-1,\CRR)$ is the space of solutions of
   the differential equation $f''=f$, or equivalently the space of
   functions of the form $f(t)=ue^t+ve^{-t}$ with $u,v\in\R$.  For
   each such function we get a homomorphism
   $\al\:\R[D]/(D^2-1)\xra{}\CRR$ given by
   \[ \al(\ov{a+bD}) = (a+bD)(ue^t+ve^{-t}) =
       (a+b)ue^t + (a-b)ve^{-t}.
   \]
 \end{itemize}
\EndDeferredSolution

\BeginDeferredSolution{ex-Z-nine-SES}{6.4}
 Recall that there is a map $\phi\:\Z_p\xra{}\Z_q$ with
 $\phi(\ov{m})=\ov{rm}$ iff $rp$ is divisible by $q$.  This is
 satisfied when $r=p=3$ and $q=9$, so $\al$ exists.  It is also
 satisfied when $p=9$ and $r=1$ and $q=3$, so $\bt$ exists.

 The elements of $\Z_3$ are $\ov{0}$, $\ov{1}$ and $\ov{2}$.  We have
 $\al(\ov{0})=\ov{0}$, $\al(\ov{1})=\ov{3}$ $\al(\ov{2})=\ov{6}$, so
 the image of $\al$ is $\{\ov{0},\ov{3},\ov{6}\}$.

 Next, the elements of $\Z_9$ are $\ov{0},\ldots,\ov{8}$.  We have
 $\bt(\ov{3})=\ov{3}$.  The $\ov{3}$ on the left hand side is
 interpreted as an element of $\Z_9$ and thus is nonzero, but the
 $\ov{3}$ on the right hand side is interpreted as an element of
 $\Z_3$ and thus is zero.  In other words, we have
 $\bt(\ov{3})=\ov{0}$.  Similarly, we have
 \begin{align*}
  \bt(\ov{0}) &= \bt(\ov{3}) = \bt(\ov{6}) = \ov{0} \\
  \bt(\ov{1}) &= \bt(\ov{4}) = \bt(\ov{7}) = \ov{1} \\
  \bt(\ov{2}) &= \bt(\ov{5}) = \bt(\ov{8}) = \ov{2}.
 \end{align*}
 Thus
 $\ker(\bt)=\{a\in\Z_9\st\bt(a)=\ov{0}\}=\{\ov{0},\ov{3},\ov{6}\}$.
 We have shown that $\ker(\bt)=\img(\al)$, so the sequence
 $\Z_3\xra{\al}\Z_9\xra{\bt}\Z_3$ is exact.
\EndDeferredSolution

\BeginDeferredSolution{ex-no-homs}{6.5}
 \begin{itemize}
  \item[(a)] It is easy to see that $A^2=3A$, so for all $v\in M_A$ we
   have $(x^2-3x)v=0$.  In particular, for $u\in\C[x]/(x^2-1)$ we have
   $(x^2-3x)\al(u)=0$.  However, we also have $(x^2-1)u=0$ and so
   $(x^2-1)\al(u)=0$.  The polynomials $x^2-3x=x(x-3)$ and
   $x^2-1=(x-1)(x+1)$ are coprime, so there exist polynomials $p(x)$
   and $q(x)$ with $p(x)(x^2-3x)+q(x)(x^2-1)=1$.  It follows that
   \[ \al(u)=p(x)(x^2-3x)\al(u) + q(x)(x^2-1)\al(x)=0+0=0. \]
   As this holds for all $u\in\C[x]/(x^2-1)$, we have $\al=0$ as
   claimed.
  \item[(b)] As $\sin''=-\sin$ and $\cos''=-\cos$ we have
   $(D^2+1)V=0$.  As $\sinh''=\sinh$ and $\cosh''=\cosh$ we have
   $(D^2-1)W=0$.  The elements $D^2+1$ and $D^2-1$ are coprime in
   $\R[D]$, so $\bt=0$ by the same argument as in part~(a).
  \item[(c)] We have $\gm(\ov{1})=(w,x,y,z)\in\Z^4$ for some
   $w,x,w,z\in\Z$.  As $4.\ov{1}=\ov{0}$ we see that
   $(4w,4x,4y,4z)=\gm(\ov{0})=(0,0,0,0)$, so $4w=4x=4y=4z=0$.  As $w$,
   $x$, $y$ and $z$ are just integers, this implies that $w=x=y=z=0$,
   so $\gm(\ov{1})=0$.  This in turn implies that
   $\gm(\ov{n})=n.\gm(\ov{1})=n.0=0$ for all $n$.
 \end{itemize}
\EndDeferredSolution

\BeginDeferredSolution{ex-count-Hom}{6.6}
 A homomorphism from $R^d$ to $M$ corresponds to a list
 $(m_1,\ldots,m_d)$ of elements of $M$.  There are $m$ possible
 choices for each entry in the list, so there are $m^d$ possible
 lists, and thus $m^d$ different homomorphisms from $R^d$ to $M$.
\EndDeferredSolution

\BeginDeferredSolution{ex-SIT}{7.1}
 Let $\pi$ be as described.  Every element of $(L+N)/N$ has the form
 $z+N$ for some $z\in L+N$.  We can write $z$ as $x+y$ for some
 $x\in L$ and $y\in N$, so $z+N=x+y+N=x+N$ (because $y+N=N$).  Thus
 every element of $(L+N)/N$ has the form $\pi(x)$ for some $x\in L$,
 which means that $\pi$ is surjective.

 Next, $\ker(\pi)$ is the set of those $x\in L$ for which $x+N=N$, or
 in other words those $x\in L$ for which we also have $x\in N$, so
 $\ker(\pi)=L\cap N$.

 The First Isomorphism Theorem now tells us that
 \[ L/(L\cap N)=L/\ker(\pi)\simeq \img(\pi) = (L+N)/N. \]
\EndDeferredSolution

\BeginDeferredSolution{ex-int-ext}{7.2}
 The homomorphism $\sg$ is an isomorphism iff it is both injective and
 surjective, or equivalently $\ker(\sg)=\{0\}$ and $\img(\sg)=M$.  We
 have $\img(\sg)=M$ iff every element of $M$ can be written in the form
 $n_0+n_1$ for some $n_0\in N_0$ and $n_1\in N_1$, or equivalently
 $M=N_0+N_1$.  Next, $\ker(\sg)$ is the set of pairs $(n,-n)$ where
 $n\in N_0$ and $-n\in N_1$.  However, we have $-n\in N_1$ iff
 $n\in N_1$, so $\ker(\sg)=\{(n,-n)\st n\in N_0\cap N_1\}$.  It
 follows that $\ker(\sg)=\{0\}$ iff $N_0\cap N_1=\{0\}$.  Thus $\sg$
 is an isomorphism iff $M=N_0+N_1$ and $N_0\cap N_1=\{0\}$, which
 means precisely that $M$ is the internal direct sum of $N_0$ and
 $N_1$.
\EndDeferredSolution

\BeginDeferredSolution{ex-no-ring-homs}{8.1}
 \begin{itemize}
  \item[(a)] Let $\al$ be a ring homomorphism from $\Z_3$ to $\Z$.  By
   applying $\al$ to the equation $\ov{1}+\ov{1}+\ov{1}=\ov{0}$ we
   obtain $\al(\ov{1})+\al(\ov{1})+\al(\ov{1})=\al(\ov{0})$ but
   $\al(\ov{1})=1$ and $\al(\ov{0})=0$ so $1+1+1=0$ in $\Z$.  This is
   clearly false, so no such $\al$ can exist.
  \item[(b)] Let $\al$ be a ring homomorphism from $\Q$ to $\Z$.  By
   applying $\al$ to the equation $\half.(1+1)=1$ we get
   $\al(\half).(\al(1)+\al(1))=\al(1)$.  We also have $\al(1)=1$ so
   $\al(\half).2=1$.  However, there is no element $x\in\Z$ with
   $x.2=1$ so this is impossible, so no such $\al$ can exist.
  \item[(c)] Let $\al$ be a ring homomorphism from $\C$ to $\R$.  By
   applying $\al$ to the equation $i^2+1=0$ we obtain
   $\al(i)^2+\al(1)=\al(0)$ or in other words $\al(i)^2+1=0$.  There
   is no element $x\in\R$ with $x^2+1=0$, so this is impossible, so no
   such $\al$ can exist.
  \item[(d)] The only reasonable example is given by $\al(z)=\ov{z}$
   (the complex conjugate of $z$).  This is a ring homomorphism
   because $\ov{z+w}=\ov{z}+\ov{w}$ and $\ov{zw}=\ov{z}\;\ov{w}$ and
   $\ov{1}=1$.  (There are some other examples defined by a bizarre
   procedure involving heavy set theory.  The above example is the
   only one that is a continuous function from $\C$ to itself.)
 \end{itemize}
\EndDeferredSolution

\BeginDeferredSolution{ex-FIT-rings}{8.2}
 \begin{itemize}
  \item[(a)] Define $\al\:\Q[x]\xra{}\R$ by $\al(f)=f(\sqrt{2})$; this
   is clearly a ring homomorphism.  As $(\sqrt{2})^2-2=0$ we have
   $x^2-2\in\ker(\al)$, so $\Q[x].(x^2-2)\sse\ker(\al)$.  Conversely,
   suppose that $f\in\ker(\al)$, so $f(\sqrt{2})=0$.  We can divide
   $f(x)$ by $x^2-2$ to get $f(x)=(x^2-2)q(x)+a+bx$ for some
   $a,b\in\Q$.  We then have
   \[ 0=f(\sqrt{2})=(\sqrt{2}^2-2)q(\sqrt{2})+a+b\sqrt{2}=a+b\sqrt{2}. \]
   If $b\neq 0$ we can deduce that $\sqrt{2}=-a/b$ which is impossible
   as $\sqrt{2}$ is irrational.  Thus, we must have $b=0$, in which
   case the equation $0=a+b\sqrt{2}$ tells us that $a=0$ also.  Thus
   $f(x)=(x^2-2)q(x)$, so $f(x)\in\Q[x].(x^2-2)$.  Thus
   $\ker(\al)=\Q[x].(x^2-2)$, so $\Q[x]/(x^2-2)\simeq\img(\al)$ (by
   the First Isomorphism Theorem for rings), and $\img(\al)$ is a
   subring of $\R$ as required.
  \item[(b)]
   \begin{itemize}
    \item[(i)] We have $\al(-5)=-5+I$, and we want to show that this
     is the same as $i+I$, or in other words that $-5-i\in I$.  By
     direct calculation we have $(-5-i)/(2+3i)=1+i$ which lies in
     $\Z[i]$, so $-5-i=(1+i)(2+3i)\in \Z[i].(2+3i)=I$ as required.

     Now suppose we have an element $a+ib+I\in\Z[i]/I$ (so
     $a,b\in\Z$).  We find that
     $\al(a-5b)=\al(a)+\al(b)\al(-5)=a+bi+I$, and it follows that
     $\al$ is surjective.
    \item[(ii)] Suppose that $n=(2+3i)(u+iv)$.  By taking norms we
     find that $n^2=N(n)=N(2+3i)N(u+iv)=13(u^2+v^2)$, so $n^2$ is
     divisible by $13$ in $\Z$.
    \item[(iii)] As $13=(2+3i)(2-3i)\in I$ we have $\al(13)=0$ so
     $13\Z\sse\ker(\al)$.  Conversely, suppose that $\al(n)=0$, so $n$
     is divisible by $2+3i$ in $\Z[i]$.  By~(ii) we see that $n^2$ is
     divisible by $13$, but $13$ is prime so $n$ itself must be
     divisible by $13$, so $n\in 13\Z$.  Thus $\ker(\al)=13\Z$ and
     $\Z_{13}=\Z/13\Z\simeq\img(\al)=\Z[i]/I$.
   \end{itemize}
 \end{itemize}
\EndDeferredSolution

\BeginDeferredSolution{ex-C-as-quotient}{8.3}
 \begin{itemize}
  \item[(a)] Define $\al\:\R[x]\xra{}\C$ by $\al(f)=f(2i)$.  This is
   clearly a ring homomorphism.  Any complex number $a+ib$ can be
   written as $\al(a+bx/2)$, so $\al$ is surjective.  Put
   \[ I=\ker(\al)= \{f\in\R[x]\st f(2i)=0\}. \]
   The First Isomorphism Theorem for rings now tells us that
   $\R[x]/I\simeq\C$, so it will be enough to show that
   $I=\R[x].(x^2+4)$.  It is clear that the polynomial $f(x)=x^2+4$
   satisfies $f(2i)=0$, so $x^2+4\in I$, so $\R[x].(x^2+4)\sse I$.

   Conversely, suppose that $g(x)\in I$, so $g(2i)=0$.  By the
   division algorithm we have $g(x)=q(x)(x^2+4)+ax+b$ for some
   polynomial $q(x)\in\R[x]$ and some $a,b\in\R$.  If we substitute
   $x=2i$ in this relation and use the fact that $g(2i)=0$ we find
   that $2ai+b=0$.  As $a$ and $b$ are real, we can conclude that
   $a=b=0$, so $g(x)=q(x)(x^2+4)$, so $g(x)\in\R[x].(x^2+4)$.  Thus
   $I=\R[x].(x^2+4)$, as required.
  \item[(b)] In $\R[x]/(x^2-4)$ the elements $\ov{x-2}$ and $\ov{x+2}$
   are nonzero, but their product is $\ov{x^2-4}=\ov{0}$.  Thus
   $\R[x]/(x^2-4)$ is not an integral domain, and thus not a field.
 \end{itemize}
\EndDeferredSolution

\BeginDeferredSolution{ex-F-nine}{8.4}
 \begin{itemize}
  \item[(a)] I claim that
   \[ R=\{\ov{0},\ov{1},\ov{2},
          \ov{i},\ov{1+i},\ov{2+i},
          \ov{2i},\ov{1+2i},\ov{2+2i}\},
   \]
   or equivalently $R=\{\ov{a+bi}\st a,b\in\{0,1,2\}\}$.  Indeed, the
   listed elements are certainly contained in $R$, and it is easy to
   see that they are all different.  Conversely, any element $x\in R$
   can be written as $x=\ov{a+ib}$ for some $a,b\in\Z$.  We can then
   write $a=3c+a'$ for some $c\in\Z$ and $a'\in\{0,1,2\}$ and
   $c\in\Z$.  Similarly we have $b=3d+b'$ for some $d\in\Z$ and
   $b'\in\{0,1,2\}$.  It follows that $(a+bi)=(a'+b'i)+3(c+di)$, so
   $x=\ov{a'+b'i}$, so $x$ is in our list.
  \item[(b)]
   Here we will just write $a+bi$ for $\ov{a+bi}$.  We have
   \begin{align*}
    u^0 &= 1 \\
    u^1 &= 1+i \\
    u^2 &= (1+i)^2 = 2i \\
    u^3 &= u^2.u = 2i(1+i) = 2i-2 = 2i+1 \\
    u^4 &= (u^2)^2 = -4 = 2 \\
    u^5 &= u^4.u = 2+2i \\
    u^6 &= u^4.u^2 = 4i = i \\
    u^7 &= u^4.u^3 = 4i+2 = i+2 \\
    u^8 &= (u^4)^2 = 4 = 1.
   \end{align*}
  \item[(c)] On comparing~(a) with~(b) we see that every nonzero
   element of $R$ is a power of $u$.  We also have $u^8=1$, so
   $u^{8-k}$ is an inverse for $u^k$, so every nonzero element of the
   ring $R$ is invertible.  This means that $R$ is a field.
  \item[(d)] If we can prove that $3$ is irreducible in $\Z[i]$, it
   will follow that $\Z[i]/3$ is a field (Proposition 10.6 in the
   notes).  It is a general fact that prime numbers of the form $4k-1$
   are irreducible in $\Z[i]$, and this obviously covers the case of
   the prime $3$.  More explicitly, if $3$ were reducible we would
   have $3=rs$ for some nonunits $r$ and $s$.  We would then have
   $9=N(3)=N(r)N(s)$, and $N(r),N(s)\neq 1$ as $r$ and $s$ are not
   units.  This means we must have $N(r)=N(s)=3$.  However, $3$ cannot
   be written as $a^2+b^2$ for any integers $a$ and $b$, so we cannot
   have $N(r)=3$, so $3$ must be irreducible after all.
 \end{itemize}
\EndDeferredSolution

\BeginDeferredSolution{ex-roots-of-unity}{9.1}
 Let $h$ be the gcd of $f$ and $g$, which we can take to be monic.
 Note that $f(1)=g(1)=0$, so both $f$ and $g$ are divisible by $x-1$,
 so $h$ is divisible by $x-1$, or equivalently $h(1)=0$.  We claim
 that this is the only root of $h$.  Indeed, suppose that $h(\zt)=0$.
 As $h$ divides both $f$ and $g$ and $h(\zt)=0$, we see that
 $f(\zt)=g(\zt)=0$, so $\zt^n=\zt^m=1$.  We are also given that $n$
 and $m$ are coprime, so $nu+mv=1$ for some integers $u$ and $v$.  We
 deduce that
 \[ \zt = \zt^1 = \zt^{nu+mv} = (\zt^n)^u (\zt^m)^v = 1^u 1^v = 1, \]
 so $\zt=1$ as claimed.  As this is the only root of $h$, we see that
 $h(x)=(x-1)^k$ for some $k>0$.  To show that $k=1$, it will suffice
 to check that $f$ and $g$ are not divisible by $(x-1)^2$, or
 equivalently that $f'(1)\neq 0\neq g'(1)$.  This is clear from the
 formulae: we have $f'(x)=nx^{n-1}$, so $f'(1)=n>0$, and similarly
 $g'(1)=m>0$.
\EndDeferredSolution

\BeginDeferredSolution{eg-p-local-gcd}{9.2}
 We can write $a=p^nr/s$ for some integer $n\geq 0$ and some integers
 $r,s$ that are not divisible by $p$.  Similarly, we can write
 $b=p^mt/u$ for some integer $m\geq 0$ and some integers
 $t,u$ that are not divisible by $p$.  If $n\leq m$ then the number
 $x=b/a=p^{m-n}ts/ru$ lies in $\Z_{(p)}$ and $b=ax$.  This says that
 $b$ is divisible by $a$, and it follows easily that $a$ is a gcd of
 $a$ and $b$.  Similarly, if $n\geq m$ then $b$ is a gcd of $a$ and
 $b$.
\EndDeferredSolution

\BeginDeferredSolution{ex-find-bases}{11.1}
 \begin{itemize}
  \item[(a)] Put $u=(1,1,0)$ and $v=(0,1,1)$ and $w=(0,0,5)$.  I claim
   that these vectors form a basis for $M_0$ over $\Z$.  Indeed, it is
   easy to see that $u$, $v$ and $w$ all lie in $M_0$.  Moreover, if
   $m=(x,y,z)\in M_0$ then $x-y+z=5t$ for some $t$, so $z=5t-x+y$, and
   one checks that
   \begin{align*}
    xu+(y-x)v+tw &= (x,x,0) + (0,y-x,y-x) + (0,0,5t) \\
                 &= (x,y,5t+y-x) = (x,y,z) = m.
   \end{align*}
   This shows that $m$ lies in the submodule generated by $u$, $v$ and
   $w$, and it is clear that $u$, $v$ and $w$ are linearly independent
   over $\Z$, so they form a basis as claimed.
  \item[(b)] Here we put $u=(2,0,0)$ and $v=(3,3,0)$ and $w=(1,1,1)$;
   these are easily seen to be elements of $M_1$.  Given an arbitrary
   element $m=(x,y,z)\in M_1$, we note that $x-y=2s$ and $y-z=3t$ for
   some integers $s,t$.  It follows that
   \begin{align*}
    su + tv + zw &= (2s,0,0) + (3t,3t,0) + (z,z,z) \\
                 &= (x-y,0,0) + (y-z,y-z,0) + (z,z,z) \\
                 &= (x,y,z) = m.
   \end{align*}
   This shows that $m$ lies in the submodule generated by $u$, $v$ and
   $w$, and it is clear that $u$, $v$ and $w$ are linearly independent
   over $\Z$, so they form a basis as claimed.
  \item[(c)] Here we put $u=(5,-2,0)$ and $v=(0,2,-3)$; these are
   easily seen to be elements of $M_2$.  Now consider an arbitrary
   element $m=(x,y,z)\in M_2$, so $6x+15y+10z=0$.  We can reduce this
   equation modulo $5$: as $6x=x\pmod{5}$ and $15y=10z=0\pmod{5}$, we
   deduce that $x=0\pmod{5}$.  Similarly, we can reduce modulo $2$ to
   show that $y=0\pmod{2}$, and reduce mod $3$ to see that
   $z=0\pmod{3}$.  We thus have $(x,y,z)=(5r,2s,3t)$ for some
   $r,s,t\in\Z$.  The equation $6x+15y+10z=0$ now gives
   $30r+30s+30t=0$, so $r+s+t=0$.  It follows that
   \begin{align*}
    ru-tv &= (5r,-2r,0) - (0,2t,-3t) \\
          &= (5r,-2(t+r),3t) \\
          &= (5r,2s,3t) = (x,y,z) = m.
   \end{align*}
   This shows that $m$ lies in the submodule generated by $u$ and $v$,
   and it is clear that $u$, $v$ and $w$ are linearly independent over
   $\Z$, so they form a basis as claimed.
 \end{itemize}
\EndDeferredSolution

\BeginDeferredSolution{ex-split-quotient}{11.2}
 We can define $\al\:R^n\xra{}R/d_1\op\ldots\op R/d_n$ by
 \[ \al(a_1,\ldots,a_n) = (a_1+Rd_1,\ldots,a_n+Rd_n). \]
 Suppose we have an element
 $(u_1,\ldots,u_n)\in R/d_1\op\ldots\op R/d_n$, so $u_i\in R/d_i$ for
 $i=1,\ldots,n$.  For each $i$ we can choose $a_i\in R$ such that
 $u_i=a_i+Rd_i$, and then $\al(a_1,\ldots,a_n)=(u_1,\ldots,u_n)$.
 Thus $\al$ is surjective, and the First Isomorphism Theorem now tells
 us that $R^n/\ker(\al)\simeq R/d_1\op\ldots\op R/d_n$.  We now need
 to determine the kernel of $\al$.  If $\al(a_1,\ldots,a_n)=0$ then
 $a_i+Rd_i$ must be the zero element of $R/d_i$ for all $i$.  This
 means that $a_i\in Rd_i$, so $a_i=b_id_i$ say.  It follows that
 $\un{a}=\sum_ia_ie_i=\sum_ib_i.(d_ie_i)\in N$.  Conversely, it is
 clear that $\al(d_ie_i)=0$ for all $i$, so that $N\sse\ker(\al)$, so
 $\ker(\al)=N$.  Thus $R^n/N\simeq R/d_1\op\ldots\op R/d_n$ as
 claimed.
\EndDeferredSolution

\BeginDeferredSolution{ex-compatible-bases}{11.3}
 One solution is as follows:
 \begin{align*}
  u_1 &= (1,1,1,1) \hspace{6em} d_1=1 \\
  u_2 &= (0,0,0,2) \hspace{6em} d_1=2 \\
  u_3 &= (0,1,1,0) \hspace{6em} d_1=4 \\
  u_4 &= (0,0,1,1) \hspace{6em} d_1=4.
 \end{align*}
 Put $v_i=d_iu_i$.  It is clear that the set $\{u_1,\ldots,u_4\}$ is
 linearly independent, as is the set $\{v_1,\ldots,v_4\}$.  It will
 thus be enough to show that the elements $u_i$ generate $F$, and the
 elements $v_i$ generate $G$.

 It is clear that the elements $u_i$ all lie in $F$.  Suppose we have
 an element $f=(w,x,y,z)\in F$, so $w+x+y+z=2t$ for some integer $t$.
 We then have
 \begin{align*}
  wu_1 + (t-w-x)u_2 + (x-w)u_3 + (y-x)u_4 &=
   (w,w,w,w)+(0,0,0,2t-2w-2x)+ \\
  &\;\; (0,x-w,x-w,0)+(0,0,y-x,y-x) \\
  &= (w,x,y,2t-w-x-y) = (w,x,y,z).
 \end{align*}
 Thus $f$ lies in the $\Z$-submodule generated by
 $\{u_1,\ldots,u_4\}$, as required.

 Next, we have
 \begin{align*}
  v_1 &= (1,1,1,1) \\
  v_2 &= (0,0,0,4) \\
  v_3 &= (0,4,4,0) \\
  v_4 &= (0,0,4,4).
 \end{align*}
 These vectors clearly lie in $G$.  Suppose we have an element
 $g=(w,x,y,z)\in G$, so $w-x=4r$, $x-y=4s$ and $y-z=4t$ for some
 integers $r,s,t$.  We then have $x=w-4r$ and $y=x-4s=w-4r-4s$ and
 $z=y-4t=w-4r-4s-4t$, so $g=(w,w-4r,w-4r-4s,w-4r-4s-4t)$.  From this
 we see directly that $g=wv_1-(t+r)v_2+rv_3+sv_4$.  Thus $f$ lies in
 the $\Z$-submodule generated by $\{u_1,\ldots,u_4\}$, as required.

 We now deduce that
 \[ \frac{F}{G}\simeq
    \frac{\Z\op\Z\op\Z\op\Z}{\Z\op2\Z\op4\Z\op4\Z}\simeq
    \Z_2\op\Z_4\op\Z_4.
 \]
\EndDeferredSolution

\BeginDeferredSolution{ex-col-red-i}{12.1}
 \begin{align*}
   \bbm x-1 & x & x+1 \\ x & 0 & x \\ x+1 & x & x-1 \ebm
   &\xra{1} \bbm -1 & x & 1 \\ x & 0 & x \\ 1 & x & -1 \ebm \\
   &\xra{2} \bbm 1 & 0 & 0 \\ -x & x^2 & 2x \\ -1 & 2x & 0 \ebm \\
   &\xra{3} \bbm 1 & 0 & 0 \\ -x & 2x & 0 \\ -1 & 0 & 2x \ebm \\
   &\xra{4} \bbm 1 & 0 & 0 \\ 0 & x & 0 \\ -1 & 0 & x \ebm.
 \end{align*}
 We will refer to the three columns as $C_1$, $C_2$ and $C_3$.  In
 step~1 we subtracted $C_2$ from $C_1$ and from $C_3$ columns.  In
 step~2 we multiplied $C_1$ by $-1$, then subtracted $xC_1$ from $C_2$
 and subtracted $C_1$ from $C_3$.  In step~3 we exchanged $C_2$ and
 $C_3$, and then subtracted $\half x C_2$ from $C_3$.  This gives us a
 matrix in column echelon form.  In step~4 we tidy up a little further
 by multiplying $C_2$ and $C_3$ by $\half$ and then addin $C_2$ to
 $C_1$.
\EndDeferredSolution

\BeginDeferredSolution{ex-gauss-i}{12.2}
 \begin{itemize}
  \item[(a)]
   In step~$1$ we add $x$ times the middle row to the top row, and
   then multiply the middle row by $-1$.  In step~$2$ we add $x$ times
   the first column to the middle column, and subtract the first
   column from the last column.  In step~$3$ we subtract $x^2$ times
   the bottom row from the top row.  Finally, in step~$4$ we add $1+x$
   times the middle column to the last column, and then move the top
   row down to the bottom.
   \begin{align*}
    \bbm x & 0 & -1 \\ -1 & x & -1 \\ 0 & -1 & x+1 \ebm &\xra{1}
    \bbm 0 & x^2 & -1-x \\ 1 &-x &1 \\ 0 & 1 & -1-x \ebm \xra{2}
    \bbm 0 & x^2 & -1-x \\ 1 &0 &0 \\ 0 & 1 & -1-x \ebm \\
    &\xra{3}
    \bbm 0&0& x^3+x^2-x-1 \\ 1&0&0 \\ 0 & 1 & -1-x \ebm \xra{4}
    \bbm 1&0&0 \\ 0&1&0 \\ 0&0&x^3+x^2-x-1 \ebm.
   \end{align*}
   This gives us a matrix in normal form.
  \item[(b)] It follows that
   \[ M\simeq \C[x]/(x^3+x^2-x-1). \]
  \item[(c)] We have $x^3+x^2-x-1=(x^2-1)(x+1)=(x-1)(x+1)^2$.  As
   $x-1$ and $x+1$ are coprime, The Chinese Remainder Theorem implies
   that
   \[ M \simeq \C[x]/(x^3+x^2-x-1) \simeq
        \C[x]/(x-1) \op \C[x]/(x+1)^2 = B(1,1) \op B(-1,2).
   \]
 \end{itemize}
\EndDeferredSolution

\BeginDeferredSolution{ex-gauss-ii}{12.3}
 \begin{itemize}
  \item[(a)]
   \begin{align*}
    \bbm x & 0 & 1 \\ 1 & x & 0 \\ 0 & 1 & x \ebm
    &\xra{1} \bbm 0 & 0 & 1 \\ 1 & x & 0 \\ -x^2 & 1 & x \ebm \\
    &\xra{2} \bbm 1 & 0 & 0 \\ 0 & 1 & x \\ 0 & -x^2 & 1 \ebm \\
    &\xra{3} \bbm 1 & 0 & 0 \\ 0 & 1 & 0 \\ 0 & -x^2 & 1+x^3 \ebm \\
    &\xra{4} \bbm 1 & 0 & 0 \\ 0 & 1 & 0 \\ 0 & 0 & 1+x^3 \ebm.
   \end{align*}
   We refer to the rows as $R_1$, $R_2$ and $R_3$ and to the columns
   as $C_1$, $C_2$ and $C_3$.

   In step~1 we subtract $xC_3$ from $C_1$.  In step~2 we move $C_3$
   to the front and then subtract $xR_1$ from $R_3$.  In step~3 we
   subtract $xC_2$ from $C_3$.  In step~4 we add $x^2R_2$ to $R_3$.
  \item[(b)]
   It follows from~(a) that $M\simeq R/1\op R/1\op R/(1+x^3)$.  As
   $R/1=\{0\}$ it follows that $M\simeq R/(1+x^3)$, so we may take
   $f(x)=1+x^3$.
 \end{itemize}
\EndDeferredSolution

\BeginDeferredSolution{ex-gauss-iii}{12.4}
 \begin{itemize}
  \item[(a)]
   \begin{align*}
             \bbm x & 0 & 0 \\ 0 & x-1 & 0 \\ 0 & 0 & x^2-x \ebm
    &\xra{1} \bbm x & x-1 & 0 \\ 0 & x-1 & 0 \\ 0 & 0 & x^2-x \ebm
     \xra{2} \bbm 1 & x-1 & 0 \\ 1-x & x-1 & 0 \\ 0 & 0 & x^2-x \ebm\\
    &\xra{3} \bbm 1 & 0 & 0 \\ 1-x & x^2-x & 0 \\ 0 & 0 & x^2-x \ebm
     \xra{4} \bbm 1 & 0 & 0 \\ 0 & x^2-x & 0 \\ 0 & 0 & x^2-x \ebm.
   \end{align*}
   In step~1 we add $R_2$ to $R_1$, and in step~2 we subtract $C_2$
   from $C_1$.  In step~3 we subtract $(x-1)C_1$ from $C_2$, and in
   step~4 we subtract $(1-x)R_1$ from $R_2$.
  \item[(b)]
   It follows that $M\simeq\C[x]/(x^2-x)\op\C[x]/(x^2-x)$.  (As usual,
   we have omitted the factor of $\C[x]/1$, because $\C[x]/1=\{0\}$.)
 \end{itemize}
\EndDeferredSolution

\BeginDeferredSolution{ex-gauss-iv}{12.5}
 We need to reduce $A$ to normal form by row and column operations.
 We start by subtracting multiples of the first column from the other
 columns to clear out the entries in the top row.  We then subtract
 the first row from each of the other rows to clear out the first
 column.  Finally, we multiply all the rows except the first one by
 $-1$.  This leaves the following matrix:
 \[ \left( \begin{array}{c|cccc}
             1 &  0 &  0 &  0 &  0 \\ \hline
             0 &  2 &  2 &  2 &  4 \\
             0 &  0 &  0 &  4 &  4 \\
             0 &  0 &  3 &  3 &  3 \\
             0 &  2 &  0 &  4 &  2 \end{array}\right)
 \]
 From here onwards, we can ignore the first row and column and just
 work with the remaining $4\tm 4$ block.  This can be reduced as
 follows:
 \[ \bbm  2 &  2 &  2 &  4 \\
          0 &  0 &  4 &  4 \\
          0 &  3 &  3 &  3 \\
          2 &  0 &  4 &  2 \ebm \xra{1}
    \bbm  1 &  2 & -1 & -1 \\
          4 &  0 &  0 &  4 \\
          3 &  0 &  3 &  3 \\
          2 &  2 &  0 &  4 \ebm \xra{2}
    \left(\begin{array}{c|ccc}
          1 &  0 &  0 &  0 \\\hline
          0 & -8 &  4 &  8 \\
          0 & -6 &  6 &  6 \\
          0 & -2 &  2 &  6 \end{array}\right)
 \]
 In step $1$ we subtracted the third row from the first row to get
 $4-3=1$ in the top right corner, then moved the last column to the
 front so as to get a $1$ in the top left corner.  In step~$2$ we
 subtracted multiples of the first column from the remaining columns
 to clear out the top row, and then cleared out the first column.

 From here onwards we can ignore the first row and column and just
 work with the remaining $3\tm 3$ block.  This can be reduced as
 follows:
 \[ \bbm -8 & 4 & 8 \\ -6 & 6 & 6 \\ -2 & 2 & 6 \ebm \xra{1}
    \bbm 2 & -2 & 6 \\ -8 & 4 & 8 \\ -6 & 6 & 6 \ebm \xra{2}
    \bbm 2 & 0 & 0 \\ 0 & -4 & -16 \\ 0 & 0 & -12 \ebm \xra{3}
    \bbm 2 & 0 & 0 \\ 0 & 4 & 0 \\ 0 & 0 & 12 \ebm.
 \]
 In step~$1$, we moved the bottom row to the top and multiplied it by
 $-1$.  In step~$2$ we added multiples of the first column to the
 other columns to clear out the top row, then cleared out the first
 column.  In step~$3$ we subtracted $4$ times the middle column from
 the last column, and then multiplied both these columns by $-1$.

 The conclusion is that $\Z^5/M\simeq \Z_2\op\Z_4\op\Z_{12}$.
\EndDeferredSolution

\BeginDeferredSolution{ex-col-red-primes}{12.6}
 \begin{align*}
  \bbm 71 & 97 & 113 & 149 \\ 1 & 1 & 1 & 1 \ebm
   &\xra{1} \bbm 71 & 26 & 42 & 7 \\ 1 & 0 & 0 & -1 \ebm \\
   &\xra{2} \bbm 1 & 5 & 0 & 7 \\ 11 & 3 & 6 & -1 \ebm \\
   &\xra{3} \bbm 1 & 0 & 0 & 0 \\ 11 & -52 & 6 & -78 \ebm \\
   &\xra{4} \bbm 1 & 0 & 0 & 0 \\ 11 & 2 & 6 & 0 \ebm \\
   &\xra{5} \bbm 1 & 0 & 0 & 0 \\ 11 & 2 & 0 & 0 \ebm.
 \end{align*}
 We refer to the columns as $C_1,\ldots,C_4$.  In step~1 we subtract
 $C_1$ from $C_2$ and $C_3$, and $2C_1$ from $C_4$.  In step~2 we
 subtract $10C_4$ from $C_1$, $3C_4$ from $C_2$ and $6C_4$ from $C_3$.
 In step~3 we subtract $5C_1$ from $C_2$, and $7C_1$ from $C_4$.  In
 step~4 we add $9C_3$ to $C_2$ and $13C_3$ to $C_4$.  Finally, in
 step~5 we subtract $3C_2$ from $C_3$.
\EndDeferredSolution

\BeginDeferredSolution{ex-det-deg}{12.7}
 \begin{itemize}
  \item[(a)] If we add one row of a matrix to another row, then the
   determinant is unchanged.  If we swap two rows, then the
   determinant just changes sign.  The only invertible elements in
   $\Z$ are $1$ and $-1$.  If we multiply a row by one of these
   invertible elements, then the determinant is either unchanged or
   just changes sign.  Similar remarks apply to column operations.
   Thus, if $B$ is obtained from $A$ by a sequence of row and column
   operations then $\det(B)=\pm\det(A)$ and so $|\det(B)|=|\det(A)|$.
  \item[(b)] Let $A$ be an $n\tm n$ matrix over $\Z$, and suppose
   that $|\det(A)|=d\neq 0$.  Let $B$ be a matrix in normal form
   obtained from $A$ by row and column operations, so $B$ has the form
   \[ B = \blockmat{D}{0_{(n-r)\tm r}}{0_{r\tm(n-r)}}{0_{(n-r)\tm(n-r)}},
   \]
   where $D$ is a diagonal matrix with nonzero diagonal entries
   $d_1,\ldots,d_r$ say.  By multiplying some rows by $-1$ if
   necessary, we may assume that $d_i>0$ for all $i$.  By part~(a) we
   have $|\det(B)|=|\det(A)|=d>0$.  If we had $r<n$ then the last
   $n-r$ columns in $B$ would be zero and we would have $\det(B)=0$,
   giving a contradiction.  We must thus have $r=n$ and so $B=D$.  As
   $D$ is a diagonal matrix we have
   $d=|\det(B)|=|\det(D)|=d_1d_2\ldots d_r$.  We also know that
   $M\simeq \Z_{d_1}\op\ldots\op\Z_{d_r}$ and thus
   $|M|=|\Z_{d_1}|\ldots|\Z_{d_r}|=d_1\ldots d_r$.  Thus $|M|=d$ as
   claimed.
  \item[(c)] Let $B$ be a matrix in normal form obtained from $A$ by
   row and column operations.  Just as in~(a) we see that $\det(B)$ is
   an invertible element of $\C[x]$ times $\det(A)$.  The invertible
   elements of $\C[x]$ are the nonzero constants, and it follows that
   the degree of $\det(B)$ is the same as the degree of $\det(A)$.

   In particular, we see that $\det(B)\neq 0$ so $B$ cannot contain
   blocks of zeros, so it must just be a diagonal matrix with nonzero
   entries $g_1(x),\ldots,g_n(x)$ say.  Let $m_i$ be the degree of
   $g_i$.  Then $\det(B)=g_1\ldots g_n$ so
   $\deg(f)=\deg(\det(B))=m_1+\ldots+m_n$.

   On the other hand, we also have
   $M\simeq\C[x]/g_1\op\ldots\op\C[x]/g_n$, so
   $\dim_\C(M)=\sum_{i=1}^n\dim_\C(\C[x]/g_i)$.  Using the division
   algorithm we see that every element of $\C[x]/g_i$ has the form
   $a_0+\ldots+a_{m_i-1}x^{m_i-1}+\C[x]g_i(x)$ for some unique list of
   coefficients $a_0,\ldots,a_{m_i-1}$.  This means that
   $\{1,x,\ldots,x^{m_i-1}\}$ is a basis for $\C[x]/g_i$, so
   $\dim_\C(\C[x]/g_i)=m_i$.  It follows that
   $\dim_\C(M)=\sum_{i=1}^n\dim_\C(\C[x]/g_i)=\sum_{i=1}^nm_i=\deg(f)$
   as claimed.
 \end{itemize}
\EndDeferredSolution

\BeginDeferredSolution{ex-hilbert-matrix}{12.8}
 We have
 \[ B = \bbm 120 & 60 & 40 \\ 60 & 40 & 30 \\ 40 & 30 & 24 \ebm. \]
 This can be put in normal form as follows:
 \begin{align*}
           \bbm 120 & 60 & 40 \\ 60 & 40 & 30 \\ 40 & 30 & 24 \ebm
  &\xra{1} \bbm 40 & 20 & 40 \\ 0 & 10 & 30 \\ -8 & 6 & 24 \ebm
   \xra{2} \bbm 60 & 20 & -40 \\ 10 & 10 & -10 \\ -2 & 6 & 0 \ebm \\
  &\xra{3} \bbm 60 & 200 & -40 \\ 10 & 40 & -10 \\ -2 & 0 & 0 \ebm
   \xra{4} \bbm 0 & 200 & 40 \\ 0 & 40 & 10 \\ 2 & 0 & 0 \ebm \\
  &\xra{5} \bbm 2 & 0 & 0 \\ 0 & 40 & 10 \\ 0 & 200 & 40 \ebm
   \xra{6} \bbm 2 & 0 & 0 \\ 0 & 10 & 40 \\ 0 & 40 & 200 \ebm  \\
  &\xra{7} \bbm 2 & 0 & 0 \\ 0 & 10 & 0 \\ 0 & 40 & 40 \ebm
   \xra{8} \bbm 2 & 0 & 0 \\ 0 & 10 & 0 \\ 0 & 0 & 40 \ebm.
 \end{align*}
 In step~1 we subtract $2C_3$ from $C_1$ and $C_3$ from $C_2$.  In
 step~2 we add $C_2$ to $C_1$.  In step~3 we add $3C_1$ to $C_2$.  In
 step~4 we multiply $C_3$ and $R_3$ by $-1$ and then subtract $30R_3$
 from $R_1$ and $5R_3$ from $R_2$.  In step~5 we swap $R_1$ and $R_3$,
 and in step~6 we swap $C_2$ and $C_3$.  In step~7 we subtract $4C_2$
 from $C_3$, and in step~8 we subtract $4R_2$ from $R_3$.

 The conclusion is that $M\simeq\Z_{2}\op\Z_{10}\op\Z_{40}$.
\EndDeferredSolution

\BeginDeferredSolution{ex-cubic-span}{12.9}
 The generators of $M$ are the columns of the matrix
 $\bbm 2&3&4&5\\8&27&64&125 \ebm$.  We can perform column operations
 to simplify this matrix as follows:
 \begin{align*}
  \bbm 2&3&4&5\\8&27&64&125 \ebm &
   \xra{} \bbm 2&1&0&1 \\ 8&19&48&109 \ebm
   \xra{} \bbm 1&2&0&1 \\ 19&8&48&109 \ebm  \\
  &\xra{} \bbm 1&0&0&0 \\ 19&-30&48&90 \ebm
   \xra{} \bbm 1&0&0&0 \\ 19&30&18&0 \ebm  \\
  &\xra{} \bbm 1&0&0&0 \\ 19&12&18&0 \ebm
   \xra{} \bbm 1&0&0&0 \\ 1&6&0&0 \ebm.
 \end{align*}
 It follows that $M$ is generated by the vectors $(1,1)$ and $(0,6)$,
 and these elements are clearly linearly independent, so they form a
 basis for $M$ over $\Z$.  Note also that $(1,1)$ and $(0,1)$ form a
 basis for $\Z^2$; using this, it is easy to see that
 $\Z^2/M\simeq\Z_6$, and so $|\Z^2/M|=6$.
\EndDeferredSolution

\BeginDeferredSolution{ex-thirty}{12.10}
 \begin{itemize}
  \item[(a)] Note that $A$ is not in normal form to start with,
   because $10$ and $15$ are not divisible by $6$.  The first step is
   to bring some numbers that are not divisible by $6$ up onto the top
   row.  It is convenient to add both the second and third rows to the
   first row.  This has the effect that the greatest common divisor of
   the numbers on the new first row is $1$.  We can then perform
   column operations until we have a $1$ in the top right corner:
   \[ \bbm 6&0&0\\0&10&0\\0&0&15\ebm \xra{}
      \bbm 6&10&15\\0&10&0\\0&0&15\ebm \xra{}
      \bbm 6&-2&3\\0&10&0\\0&0&15\ebm \xra{}
      \bbm 6&-2&1\\0&10&10\\0&0&15\ebm.
   \]
   We then subtract multiples of the first row from the other rows, to
   clear out the entries underneath the $1$.  After that, we perform
   column operations to clear out the entries to the left of the $1$.
   We then exchange the first and last columns to put the $1$ in the
   top left corner.
   \[ \bbm 6&-2&1\\0&10&10\\0&0&15\ebm \xra{}
      \bbm 6&-2&1\\-60&30&0\\-90&30&0\ebm \xra{}
      \bbm 0&0&1\\-60&30&0\\-90&30&0\ebm \xra{}
      \bbm 1&0&0\\0&30&-60\\0&30&-90\ebm.
   \]
   We now add twice the middle column to the last column, subtract the
   middle row from the last row, and multiply the last row by $-1$:
   \[ \bbm 1&0&0\\0&30&-60\\0&30&-90\ebm \xra{}
      \bbm 1&0&0\\0&30&0\\0&30&-30\ebm \xra{}
      \bbm 1&0&0\\0&30&0\\0&0&-30\ebm \xra{}
      \bbm 1&0&0\\0&30&0\\0&0&30\ebm.
   \]
   The last matrix (which we will call $B$) is in normal form.
  \item[(b)]
   As $B$ is obtained from $A$ by row and column operations, the
   quotient of $\Z^3$ by the span of the columns of $A$ is isomorphic
   to the quotient of $\Z^3$ by the span of the columns of $B$.  In
   other words,
   \[ \frac{\Z\op\Z\op\Z}{6\Z\op 10\Z\op 15\Z} \simeq
      \frac{\Z\op\Z\op\Z}{1\Z\op 30\Z\op 30\Z},
   \]
   or equivalently $\Z_6\op\Z_{10}\op\Z_{15}\simeq\Z_{30}\op\Z_{30}$.
   (We have omitted the factor of $\Z_1$, because $\Z_1=\{0\}$.)
  \item[(c)] The Chinese Remainder Theorem implies that
   \begin{align*}
    \Z_6    &\simeq \Z_2\op\Z_3 \\
    \Z_{10} &\simeq \Z_2\op\Z_5 \\
    \Z_{15} &\simeq \Z_3\op\Z_5 \\
    \Z_{30} &\simeq \Z_2\op\Z_3\op\Z_5,
   \end{align*}
   so
   \begin{align*}
    \Z_6\op\Z_{10}\op\Z_{15} &\simeq
      (\Z_2\op\Z_3)\op(\Z_2\op\Z_5)\op(\Z_3\op\Z_5) \\
    &\simeq (\Z_2\op\Z_3\op\Z_5)\op(\Z_2\op\Z_3\op\Z_5) \\
    &\simeq \Z_{30}\op\Z_{30}.
   \end{align*}
 \end{itemize}
\EndDeferredSolution

\BeginDeferredSolution{eg-decompose-i}{13.1}
 We have
 \begin{align*}
  x^4 - x^2 &= (x-1)(x+1)x^2 \\
  x^4-2x^2+1 &= (x-1)^2(x+1)^2
 \end{align*}
 so
 \begin{align*}
  \C[x]/(x^4-x^2) &\simeq \C[x]/(x-1) \op \C[x]/(x+1) \op \C[x]/x^2 \\
  \C[x]/(x^4-2x^2+1) &\simeq \C[x]/(x-1)^2 \op \C[x]/(x+1)^2 \\
  M &\simeq \C[x]/(x-1) \op \C[x]/(x-1)^2 \op \\
    &\hphantom{\simeq} \C[x]/(x+1) \op \C[x]/(x+1)^2 \op \\
    &\hphantom{\simeq} \C[x]x^2 \\
    &= B(1,1) \op B(1,2) \op B(-1,1) \op B(-1,2) \op B(0,2).
 \end{align*}
\EndDeferredSolution

\BeginDeferredSolution{ex-FpkM-calc}{13.2}
 The elements of $M$ have the form $(\ov{a},\ov{b},\ov{c})$ with
 $\ov{a}\in\Z_2=\{\ov{0},\ov{1}\}$ and
 $\ov{b}\in\Z_4=\{\ov{0},\ov{1},\ov{2},\ov{3}\}$ and
 $\ov{c}\in\Z_9=\{\ov{0},\ov{1},\ldots,\ov{8}\}$.  We have
 \begin{align*}
   F_2^1(M) &=
    \{(\ov{a},\ov{b},\ov{c}) \st
      2\ov{a}=\ov{0}\;,\;2\ov{b}=\ov{0}\;,\;2\ov{c}=\ov{0} \} \\
   &= \{(\ov{a},\ov{b},\ov{c}) \st 2 | 2a\;,\; 4|2b \text{ and } 9|2c\}.
 \end{align*}
 Clearly $2$ always divides $2a$, and $4$ divides $2b$ iff $b$ is
 even, and $9$ divides $2c$ iff $9$ divides $c$.  Thus $\ov{a}$ can be
 either element of $\Z_2$, $\ov{b}$ can be $\ov{0}$ or $\ov{2}$, and
 $\ov{c}$ must be $\ov{0}$.  Thus
 \[ F_2^1(M) = \{ (\ov{0},\ov{0},\ov{0}),
                  (\ov{0},\ov{2},\ov{0}),
                  (\ov{1},\ov{0},\ov{0}),
                  (\ov{1},\ov{2},\ov{0}) \}.
 \]
 The elements of $F_2^2(M)$ are the elements of $F_2^1(M)$ that have
 the form $2m$ for some $m\in M$.  Thus
 \[ F_2^2(M) = \{ (\ov{0},\ov{0},\ov{0}), (\ov{0},\ov{2},\ov{0}) \}. \]
 By similar arguments we have
 \[ F_3^1(M) = \{ (\ov{0},\ov{0},\ov{0}),
                  (\ov{0},\ov{0},\ov{3}),
                  (\ov{0},\ov{0},\ov{6}) \}.
 \]
\EndDeferredSolution

\BeginDeferredSolution{ex-classify-CCV}{13.3}
 \begin{itemize}
  \item[(a)] The prime factorisation of $225$ is $3^25^2$.  Thus any
   group of order $225$ is the direct sum of a group of order $3^2=9$
   and a group of order $5^2=25$.  The $3$-primary part could be
   $\Z_9$ or $\Z_3\op\Z_3$ , and the $5$-primary part could be
   $\Z_{25}$ or $\Z_5\op\Z_5$.  The possibilities are:
   \begin{align*}
    M_1 &= \Z_3\op\Z_3\op\Z_5\op\Z_5 \\
    M_2 &= \Z_9\op\Z_5\op\Z_5 \\
    M_3 &= \Z_3\op\Z_3\op\Z_{25} \\
    M_4 &= \Z_9\op\Z_{25}.
   \end{align*}
  \item[(b)] As $9$ and $25$ are coprime, The Chinese Remainder
   Theorem says that $\Z_{225}=\Z/(9\tm 25)\simeq\Z_9\op\Z_{25}=M_4$.
  \item[(c)] Only the groups $M_3$ and $M_4$ have any elements of
   order $25$.  The group $M_4$ has only $3$ elements satisfying
   $3x=0$, so we must have $M\simeq M_3$.
 \end{itemize}
\EndDeferredSolution

\BeginDeferredSolution{ex-ten-thousand}{13.4}
 \begin{itemize}
  \item[(a)] Any Abelian group $M$ of order $p^4$ can be written as a
   direct sum of groups of the form $\Z_{p^k}$ with $1\leq k\leq 4$.
   If $M\simeq\Z_{p^{k_1}}\op\ldots\Z_{p^{k_r}}$ then
   \[ p^4=|M|=p^{k_1}\tm\ldots\tm p^{k_r}=p^{k_1+\ldots+k_r}, \]
   so $k_1+\ldots+k_r=4$.  As each $k_i$ is at least $1$ this means
   that $r\leq 4$.  Using this and some trial and error we see that
   the possibilities are as follows:
   \begin{align*}
    M_1 &= \Z_p\op\Z_p\op\Z_p\op\Z_p \\
    M_2 &= \Z_p\op\Z_p\op\Z_{p^2} \\
    M_3 &= \Z_p\op\Z_{p^3} \\
    M_4 &= \Z_{p^2}\op\Z_{p^2} \\
    M_5 &= \Z_{p^4}.
   \end{align*}
  \item[(b)] In the usual notation, the question tells us that
   $f^1_p(M)=3$.  It is a standard fact that $f^1_p(\Z_{p^k})=1$ for
   all $k\geq 1$, and that $f^1_p(A\op B)=f^1_p(A)+f^1_p(B)$.  Using
   this, we find that
   \begin{align*}
    f^1_p(M_1) &= 4 \\
    f^1_p(M_2) &= 3 \\
    f^1_p(M_3) &= 2 \\
    f^1_p(M_4) &= 2 \\
    f^1_p(M_5) &= 1.
   \end{align*}
   The only possibility is thus that $M\simeq M_2$.
  \item[(c)] We have $10000=2^4\tm 5^4$.  As $2$ and $5$ are coprime,
   any Abelian group of order $2^4\tm 5^4$ is the direct sum of a
   group of order $2^4$ and a group of order $5^4$, in a unique way.
   By part~(a), there are $5$ possibilities for the $2$-primary part
   and $5$ possibilities for the $5$-primary part, giving $5\tm 5=25$
   possible groups of order $10000$.
 \end{itemize}
\EndDeferredSolution

\BeginDeferredSolution{ex-p-fifth}{13.5}
 \begin{itemize}
  \item[(a)]
   The possibilities are as follows:
   \begin{align*}
    M_1 &= \Z_p\op\Z_p\op\Z_p\op\Z_p\op\Z_p \\
    M_2 &= \Z_{p^2}\op\Z_p\op\Z_p\op\Z_p \\
    M_3 &= \Z_{p^2}\op\Z_{p^2}\op\Z_p \\
    M_4 &= \Z_{p^3}\op\Z_p\op\Z_p \\
    M_5 &= \Z_{p^3}\op\Z_{p^2} \\
    M_5 &= \Z_{p^4}\op\Z_p \\
    M_6 &= \Z_{p^5}.
   \end{align*}
  \item[(b)] Take $p=2$.  The groups $M_4,\ldots,M_6$ all contain
   elements of order $p^3=8$, but all elements in $M$ have $4m=0$, so
   we must have $M\simeq M_i$ for some $i\leq 3$.  By assumption we
   have $|F_2^1(M)|=8=2^3$ so $f_2^1(M)=\dim_{\Z_2}F_2^1(M)=3$.
   However, we know that $f_p^k(\Z_{p^j})=1$ for all $j>0$ so
   $f_p^1(M_1)=5$, $f_p^1(M_2)=4$ and $f_p^1(M_3)=3$.  We must
   therefore have $M\simeq M_3$.
 \end{itemize}
\EndDeferredSolution

\BeginDeferredSolution{ex-diffop-basis}{13.6}
 First note that $(D-1)D(D+1)f=(D^3-D)f=f'''-f'=0$.  Suppose we take
 $e_{-1}=a_{-1}D(D+1)$ for some $a_{-1}\in\R$.  Then
 $(D-1)e_{-1}f=a_{-1}(D-1)D(D+1)f=0$, as required.  Similarly, if we
 take $e_0=a_0(D-1)(D+1)$ and $e_1=a_1(D-1)D$ then we will have
 $De_0f=0$ and $(D+1)e_1f=0$.

 If we can arrange that $e_{-1}+e_0+e_1=1$ then the remaining
 condition $f=e_{-1}f+e_0f+e_1f$ will clearly also be satisfied.  We
 have
 \begin{align*}
  e_{-1}+e_0+e_1 &= a_{-1}(D^2+D) + a_0(D^2-1) + a_1(D^2-D) \\
   &= -a_0 +  (a_{-1}-a_1)D + (a_{-1}+a_0+a_1)D^2.
 \end{align*}
 For this to equal $1$, we must have $a_0=-1$ and $a_1=a_{-1}$ and
 $a_{-1}+a_0+a_1=0$, so $a_1=a_{-1}=1/2$.  Thus
 \begin{align*}
  e_{-1} &= (D^2+D)/2 \\
  e_0    &= 1-D^2 \\
  e_1    &= (D^2-D)/2.
 \end{align*}
\EndDeferredSolution

\BeginDeferredSolution{ex-diffop-solve}{13.7}
 \begin{itemize}
  \item[(a)] $p(D)=D^3-D$.
  \item[(b)] Note that $p(D)=(D-1)(D+1)D$ and all the factors are
   coprime to each other and $p(D)M=\{0\}$.  It follows that if we put
   \begin{align*}
    M_0 &= \{f\in\CRR\st (D-1)f=0\} = \{f \st f'=f\} \\
    M_1 &= \{f\in\CRR\st (D+1)f=0\} = \{f \st f'=-f\}\\
    M_2 &= \{f\in\CRR\st Df=0\} = \{f \st f'=0\}
   \end{align*}
   then $M=M_0\op M_1\op M_2$.
  \item[(c)] By standard methods we have
   \[
    M_0 = \{ae^t\st a\in\R\} \hspace{4em}
    M_1 = \{be^{-t}\st b\in\R\} \hspace{4em}
    M_2 = \R.
   \]
   As $M=M_0\op M_1\op M_2$, any function $f$ satisfying $f'''=f'$ can
   be written as $f=f_0+f_1+f_2$ with $f_i\in M_i$, or equivalently as
   $ae^t+be^{-t}+c$ for some $a,b,c\in\R$.
 \end{itemize}
\EndDeferredSolution

\BeginDeferredSolution{ex-diffop-torsion}{13.8}
 \begin{itemize}
  \item[(a)] We have
   $f'(t)=p'(t)e^{\lm t}+\lm p(t)e^{\lm t}=p'(t)e^{\lm t}+\lm f(t)$.
   Thus $((D-\lm)f)(t)=p'(t)e^{\lm t}$.  It follows inductively that
   $((D-\lm)^kf)(t)=p^{(k)}(t)e^{\lm t}$ for all $k\geq 0$.  If $p$ is
   a polynomial of degree $d$ then $p^{(k)}$ is a polynomial of degree
   $d-k$ for $k=1,\ldots,d$, and $p^{(k)}=0$ for $k>d$.  It follows
   that $(D-\lm)^kf=0$ for $k>d$.  This means that $f$ is a torsion
   element of $W_\lm$.  This holds for every element of $W_\lm$, so
   $W_\lm$ is a torsion module.
  \item[(b)] We have
   \[ g'(t)=f'(t)e^{-\lm t}-\lm f(t)e^{-\lm t}=((D-\lm)f)(t)e^{-\lm t}.
   \]
   By extending this inductively, we find that
   $g^{(k)}(t)=((D-\lm)^kf)(t)e^{-\lm t}=0$, or in other words
   $D^kg=0$.
  \item[(c)] In the case $k=1$ we have $Dg=0$ so $g$ is constant and
   so certainly polynomial of degree less than $1$.

   Suppose we have shown that all functions with $D^{k-1}g=0$ are
   polynomial of degree less than $k-1$.  If $D^kf=0$ then
   $D^{k-1}f'=0$ so $f'$ is polynomial, say
   $f'(t)=a_0+\ldots+a_{k-2}t^{k-2}$.  Put
   \[ F(t)=f(t)-f(0)=\int_{s=0}^t f(t) =
       a_0t+\ldots+a_{k-2}t^{k-1}/(k-1),
   \]
   which is clearly a polynomial of degree less than $k$.  As $f(0)$
   is just a constant we deduce that $f(t)=F(t)+f(0)$ is also a
   polynomial of degree less than $k$.
  \item[(d)] If $(D-\lm)^kf=0$ then the function $g(t)=f(t)e^{-\lm t}$
   satisfies $D^kg=0$, so $g$ is a polynomial, so $f(t)=g(t)e^{\lm t}$
   is an element of $W_\lm$.
  \item[(e)] Suppose that $p(D)f=0$.  We can factor $p(D)$ as
   $u(D-\lm_1)^{k_1}\ldots(D-\lm_r)^{k_r}$ for some nonzero constant
   $u$, where the $\lm_i$'s are all different and the $k_i$'s are all
   strictly positive.  It follows that the terms $(x-\lm_i)^{k_i}$ are
   all coprime to each other.  If we put $V=\{f\st p(D)f=0\}$ and
   $V_i=\{f\st (D-\lm_i)^{k_i}f=0\}$, it follows that
   $V=V_1\op\ldots\op V_r$.  However, we know from~(d) that
   $V_i\sse W_{\lm_i}$.  Thus
   \[ f \in V_1 + \ldots + V_r \sse W_{\lm_1} + \ldots + W_{\lm_r}. \]
   It follows immediately from the definitions that $f\in W$.
  \item[(f)] Suppose that $f\in\tors(\CRC)$.  Then $p(D)f=0$ for some
   nonzero element $p(D)\in\C[D]$, so~(e) tells us that $f\in W$.
   Conversely, if $f\in W$ then we can write
   $f(t)=p_1(t)e^{\lm_1t}+\ldots+p_r(t)e^{\lm_rt}$ for some
   $\lm_1,\ldots,\lm_r\in\C$ and polynomials
   $p_1,\ldots,p_r\in\C[t]$.  If we put $f_i(t)=p_i(t)e^{\lm_it}$ then
   $f_i\in W_{\lm_i}$, so $f_i$ is a torsion element by~(a).  Thus
   $f=f_1+\ldots+f_r$ is a sum of torsion elements and thus is a
   torsion element, as claimed.  More concretely, if the degree of
   $p_i$ is $k_i-1$ then the element
   $q(D)=(D-\lm_1)^{k_1}\ldots(D-\lm_r)^{k_r}$ is nonzero and
   satisfies $q(D)f=0$.
 \end{itemize}
\EndDeferredSolution

\BeginDeferredSolution{ex-matrix-monomials}{13.9}
 \begin{itemize}
  \item[(a)]
   In step $1$ we subtract $x^2$ times the last column from the first
   column, and subtract $x$ times the last column from the second
   column.  In step $2$ we swap the first and last columns.  In step
   $3$ we subtract the middle row from the bottom row, and then
   subtract the top row from the middle row.  Finally, we swap the
   middle row and the bottom row.
   \begin{align*}
    \bbm  x^3 & x^2 & x \\ x & x^2 & x \\ x & x & x \ebm &\xra{1}
      \bbm  0 & 0 & x \\ x-x^3&0&x \\ x-x^3&x-x^2&x \ebm \xra{2}
      \bbm  x & 0 & 0 \\ x&0&x-x^3 \\ x&x-x^2&x-x^3 \ebm \\
      &\xra{3}
      \bbm  x & 0 & 0 \\ 0&0&x-x^3 \\ 0&x-x^2&0 \ebm \xra{4}
      \bbm  x & 0 & 0 \\ 0&x-x^2&0 \\ 0&0&x-x^3 \ebm.
   \end{align*}
   Note that $x-x^2=x(1-x)$ and $x-x^3=x(1-x)(1+x)$, so
   $x|(x-x^2)|(x-x^3)$, so the last matrix is in normal form.
  \item[(b)] It follows that
   \[ M\simeq \C[x]/x\op \C[x]/(x-x^2) \op \C[x]/(x-x^3). \]
  \item[(c)] The Chinese Remainder Theorem implies that
   \begin{align*}
    \C[x]/(x-x^2) &\simeq \C[x]/x \op \C[x]/(x-1) \\
                  &\simeq B(0,1) \op B(1,1) \\
    \C[x]/(x-x^3) &\simeq \C[x]/x \op \C[x]/(x-1) \op \C[x]/(x+1) \\
                  &\simeq B(0,1) \op B(1,1) \op B(-1,1).
   \end{align*}
   Thus
   \begin{align*}
    M &\simeq \C[x]/x\op \C[x]/(x-x^2) \op \C[x]/(x-x^3) \\
      &\simeq B(0,1) \op (B(0,1)\op B(1,1)) \op
              (B(0,1) \op B(1,1) \op B(-1,1)) \\
      &\simeq B(0,1)^3 \op B(1,1)^2 \op B(-1,1).
   \end{align*}
 \end{itemize}
\EndDeferredSolution

\BeginDeferredSolution{ex-antisymmetric}{14.1}
 The characteristic polynomial is
 \begin{align*}
  \left|\begin{array}{ccc}
   t & -a & -b \\ a & t & -c \\ b & c & t
  \end{array}\right| &=
  t      \left|\begin{array}{cc} t & -c \\ c & t \end{array}\right|
  - (-a) \left|\begin{array}{cc} a & -c \\ b & t \end{array}\right|
  + (-b) \left|\begin{array}{cc} a &  t \\ b & c \end{array}\right| \\
  &= (t^3 + c^2t) + (a^2t + abc) + (-abc + b^2t) \\
  &= t^3 + r^2 t = t (t+ir)(t-ir).
 \end{align*}
 If $r\neq 0$ then the three roots of $\chr(A)$ are distinct.  It
 follows easily that the Jordan normal form is
 $J(0,1)\op J(ir,1)\op J(-ir,1)$ and thus that
 \[ M_A\simeq B(0,1)\op B(ir,1)\op B(-ir,1) =
    M_0\op M_{ir} \op M_{-ir}.
 \]
 In the exceptional case where $r=0$ we must have $a=b=c=0$ so $A$ is
 just the zero matrix and it is clear that
 $M_A\simeq M_0\op M_0\op M_0$.
\EndDeferredSolution

\BeginDeferredSolution{ex-circulant}{14.2}
 \begin{itemize}
  \item[(a)] The characteristic polynomial is by definition the
   determinant of the first matrix below.  The second matrix is
   obtained by adding the second and third rows to the top row, and
   the third matrix is obtained by subtracting the first column from
   each of the other columns.  It is a standard fact that these
   operations do not change the determinant.
   \[ \bbm x-u & -v & -w \\ -v & x-w & -u \\ -w & -u & x-v \ebm \xra{}
      \bbm x-a &x-a &x-a \\ -v & x-w & -u \\ -w & -u & x-v \ebm \xra{}
      \bbm x-a &0 &0 \\ -v & x+(v-w) & v-u \\ -w & w-u & x-(v-w)\ebm.
   \]
   The determinant of the last matrix can be expanded out directly to
   give
   \begin{align*}
    (x-a)(x^2-(v-w)^2-(w-u)(v-u)) &=
      (x-a)(x^2-v^2-w^2+2vw-vw+uv+uw-u^2) \\
     &= (x-a)(x^2-u^2-v^2-w^2+uv+vw+wu).
   \end{align*}
   On the other hand, we have
   \begin{align*}
    b &= ((u-v)^2+(v-w)^2+(w-u)^2)/2 \\
      &= (u^2+v^2-2uv + v^2+w^-2vw + w^2+u^2-2uw)/2 \\
      &= (2u^2+2v^2+2w^2-2uv-2vw-2wu)/2 \\
      &= u^2+v^2+w^2-uv-vw-wu.
   \end{align*}
   Using this, we can rewrite the characteristic polynomial as
   $(x-a)(x^2-b)$.
  \item[(b)] Suppose that the numbers $u$, $v$ and $w$ are not all the
   same, and that $uv+vw+wu\neq 0$.  The characteristic polynomial of
   $A$ is $(x-a)(x-\sqrt{b})(x+\sqrt{b})$; if we can show that the
   three roots are distinct, it will follows that this is the same as
   the minimal polynomial.

   As $u$, $v$ and $w$ are not all the same, at least one of the
   numbers $u-v$, $v-w$ and $w-u$ is nonzero.  Thus $(u-v)^2$,
   $(v-w)^2$ and $(w-u)^2$ are all nonnegative, and one of them is
   strictly positive, so the number $b=((u-v)^2+(v-w)^2+(w-u)^2)/2$ is
   strictly positive.  This means that $\sqrt{b}>0$, so
   $\sqrt{b}\neq-\sqrt{b}$.

   We also have
   \[ a^2-b=(u+v+w)^2-(u^2+v^2+w^2-uv-vw-wu)=3(uv+vw+wu)\neq 0, \]
   so $a\neq\pm\sqrt{b}$.  Thus, the three roots are all different, as
   required.

  \item[(c)] If $uv+vw+wu=0$ we see from the above equation that
   $a^2=b$, so the characteristic polynomial is
   $(x-a)(x^2-a^2)=(x-a)^2(x+a)$.  By assumption we have $a\neq 0$ so
   $x-a$ and $x+a$ are coprime.  We know from the general theory that
   the minimal polynomial divides the characteristic polynomial and
   has precisely the same roots, so it is either $(x-a)(x+a)=x^2-a^2$
   or $(x-a)(x+a)^2$.  By direct calculation one finds that
   \[ A^2-a^2I = (uv+vw+wu)\bbm -2&1&1\\1&-2&1\\1&1&-2 \ebm, \]
   which is the zero matrix because $uv+vw+wu=0$.  Thus the minimal
   polynomial is $x^2-a^2$, as claimed.

  \item[(d)] If $u=-13$, $v=11$ and $w=2$ then $a=-13+11+2=0$ and
   $b=((-13-11)^2+(11-2)^2+(2-(-13))^2)/2=441=21^2$.  The
   characteristic polynomial is $x(x^2-441)=x(x-21)(x+21)$.  The three
   roots are clearly distinct, so this is also the minimal polynomial.

  \item[(e)] If $u=-2$, $v=3$ and $w=6$ then $a=7$ and
   $uv+vw+wu=-6+18-12=0$ so part~(c) applies and the minimal polynomial
   is $x^2-49$.
 \end{itemize}
\EndDeferredSolution

\BeginDeferredSolution{ex-JNF-i}{14.3}
 \begin{itemize}
  \item[(a)] The characteristic polynomial is the determinant of the
   matrix
   \[ tI - A =
       \bbm 1+t&-1&-1&1 \\ 0&1+t&0&-1 \\ 0&0&1+t&1 \\ 0&0&0&1+t
       \ebm.
   \]
   As this is an upper-triangular matrix, the determinant is just the
   product of the diagonal entries, which is $(1+t)^4$.
  \item[(b)] The minimal polynomial is a divisor of the characteristic
   polynomial, so it must by $(1+t)^k$ for some integer $k\leq 4$.
   The matrix $I+A$ is clearly nonzero, but
   \[ (I+A)^2 =
      \bbm 0&1&1&-1 \\ 0&0&0&1 \\ 0&0&0&-1 \\ 0&0&0&0 \ebm
      \bbm 0&1&1&-1 \\ 0&0&0&1 \\ 0&0&0&-1 \\ 0&0&0&0 \ebm =
      \bbm 0&0&0&0 \\ 0&0&0&0 \\ 0&0&0&0 \\ 0&0&0&0 \ebm.
   \]
   It follows that the minimal polynomial of $A$ is $(t+1)^2$.
  \item[(c)] The first two rows of $I+A$ are clearly linearly
   independent, but the third row is minus the second, and the last
   row is zero.  Thus, the rank is two.
  \item[(d)] If $M_A$ is a direct sum of modules of the form
   $B(\lm,k)$, then the numbers $\lm$ must be roots of the minimal
   polynomial of multiplicity at least $k$.  The only root of the
   minimal polynomial is $-1$, which has multiplicity $2$.  It follows
   that the only possible summands in $M_A$ are $B(-1,1)$ (which has
   dimension $1$ over $\C$) and $B(-1,2)$ (which has dimension $2$
   over $\C$).  As $x+1$ has rank $0$ on $B(-1,1)$ and rank $1$ on
   $B(-1,2)$, we see that the number of copies of $B(-1,2)$ is the
   rank of $x+1$ on $M_A$, or in other words the rank of $A+I$, which
   is $2$.  The two copies of $B(-1,2)$ have total dimension $4$,
   which is the dimension of $M_A$, so there cannot be any copies of
   $B(-1,1)$ and we have $M_A\simeq B(-1,2)\op B(-1,2)$.
 \end{itemize}
\EndDeferredSolution

\BeginDeferredSolution{ex-JNF-ii}{14.4}
 \begin{itemize}
  \item[(a)] The characteristic polynomial is the determinant of the
   matrix
   \[ tI - A =
       \bbm t-1&-2&-3&-4 \\ 0&t-1&0&-5 \\ 0&0&t-1&-6 \\ 0&0&0&t-1
       \ebm.
   \]
   As this is an upper-triangular matrix, the determinant is just the
   product of the diagonal entries, which is $(t-1)^4$.
  \item[(b)] The minimal polynomial is a divisor of the characteristic
   polynomial, so it must be $(t-1)^k$ for some integer $k\leq 4$.
   We have
   \begin{align*}
    (A-I)^2 &=
     \bbm 0&2&3&4 \\ 0&0&0&5 \\ 0&0&0&6 \\ 0&0&0&0 \ebm
     \bbm 0&2&3&4 \\ 0&0&0&5 \\ 0&0&0&6 \\ 0&0&0&0 \ebm =
     \bbm 0&0&0&28 \\ 0&0&0&0 \\ 0&0&0&0 \\ 0&0&0&0 \ebm \neq 0 \\
    (A-I)^3 &=
     \bbm 0&0&0&28 \\ 0&0&0&0 \\ 0&0&0&0 \\ 0&0&0&0 \ebm
     \bbm 0&2&3&4 \\ 0&0&0&5 \\ 0&0&0&6 \\ 0&0&0&0 \ebm =
     \bbm 0&0&0&0 \\ 0&0&0&0 \\ 0&0&0&0 \\ 0&0&0&0 \ebm,
   \end{align*}
   and it follows that the minimal polynomial is $(t-1)^3$.
  \item[(c)] As $1$ is the only root of the minimal polynomial, the
   module $M_A$ is isomorphic to a direct sum of modules of the form
   $B(1,k)$.  As $1$ is a triple root of the minimal polynomial, we
   must have at least one summand of the form $B(1,3)$.  This has
   dimension $3$ and $M_A$ has dimension $4$ so there is only one
   dimension left over, so the other summand must be $B(1,1)$.  We
   thus have $M_A\simeq B(1,3)\op B(1,1)$.
  \item[(d)] We read off from this that $A$ is conjugate to
   $J(1,3)\op J(1,1)$, which is the following matrix:
   \[ \bbm 1&0&0&0\\1&1&0&0\\0&1&1&0\\0&0&0&1 \ebm. \]
 \end{itemize}
\EndDeferredSolution

\BeginDeferredSolution{ex-JNF-iii}{14.5}
 \begin{itemize}
  \item[(a)] The characteristic polynomial is the determinant of the
   matrix
   \[ tI - A =
       \bbm t+1 & 0   & -1  & 0 \\
            0   & t-1 & -2  & 0 \\
            0   & 0   & t+1 & 0 \\
            0   & -1  & -1  & t-1
       \ebm.
   \]
   Expanding along the top row and ignoring the $0$'s we get
   \[ \det(tI-A) =
       (t+1)\left| \begin{array}{ccc}
             t-1 & -2 & 0 \\ 0 & t+1 & 0 \\ -1 & -1 & t-1
            \end{array} \right| +
       (-1) \left| \begin{array}{ccc}
             0 & t-1 & 0 \\ 0 & 0 & 0 \\ 0 & -1  & t-1
            \end{array} \right|
   \]
   The second term here is easily seen to be $0$.  For the first term,
   we have
   \begin{align*}
    \left| \begin{array}{ccc}
     t-1 & -2 & 0 \\ 0 & t+1 & 0 \\ -1 & -1 & t-1
    \end{array} \right| &=
    (t-1) \left|\begin{array}{cc}t+1&0\\-1&t-1\end{array}\right| -
    (-2) \left|\begin{array}{cc}0&0\\-1&t-1\end{array}\right| \\
    &= (t-1)^2(t+1)-(-2)(0) = (t-1)^2(t+1).
   \end{align*}
   It follows that $\chr(A)(t)=(t-1)^2(t+1)^2$.

   A slightly quicker approach is possible if you know that
   determinants can be expanded along any row or column, not just
   along the top row.  We then have
   \begin{align*}
       \left|\begin{array}{cccc}
        t+1 & 0   & -1  & 0 \\
        0   & t-1 & -2  & 0 \\
        0   & 0   & t+1 & 0 \\
        0   & -1  & -1  & t-1
      \end{array}\right| &\stackrel{1}{=}
       (t+1)\left| \begin{array}{ccc}
             t-1 & -2 & 0 \\ 0 & t+1 & 0 \\ -1 & -1 & t-1
        \end{array} \right| \\ & \stackrel{2}{=}
       (t+1)(t-1)\left| \begin{array}{cc}
             t+1 & 0 \\ -1 & t-1
            \end{array} \right| \\
       &= (t+1)^2(t-1)^2,
   \end{align*}
   where in step~$1$ we have expanded using the first column, and in
   step~$2$ we have expanded using the third column.

  \item[(b)] We have
   \[ A+I = \bbm 0&0&1&0\\0&2&2&0\\0&0&0&0\\0&1&1&2 \ebm \hspace{4em}
      A-I = \bbm -2&0&1&0\\0&0&2&0\\0&0&-2&0\\0&1&1&0\ebm.
   \]
   In $A+I$ the nonzero rows are clearly linearly independent, so the
   rank is three.  In $A-I$ the middle two rows are multiples of each
   other but the first, second and fourth rows are linearly
   independent, so the rank is again three.

  \item[(c)] It follows from~(a) that the JNF of $A$ contains only
   blocks of type $J(1,k)$ or $J(-1,k)$.  Theorem~14.7(b) tells us
   that the number of blocks of type $J(1,k)$ is $4-\rank(A-I)$, which
   is equal to $1$ by~(b).  We thus have only one block of type
   $J(1,k)$, which has to contribute a factor $(t-1)^2$ in the
   characteristic polynomial, so it must be $J(1,2)$.  Similarly, we
   have only one block of type $J(-1,k)$ and it must be $J(-1,2)$.
   Thus the JNF of $A$ is $J(1,2)\op J(-1,2)$.

  \item[(d)] It follows from~(c) that
   \[ M_A\simeq B(1,2)\op B(-1,2) =
       \C[x]/(x-1)^2\op\C[x]/(x+1)^2\simeq \C[x]/((x-1)^2(x+1)^2).
   \]
   (The last equality uses the Chinese Remainder Theorem, which is
   valid because $(x-1)^2$ and $(x+1)^2$ are coprime.)  It follows
   that $M_A$ is cyclic, as claimed.
 \end{itemize}
\EndDeferredSolution

\BeginDeferredSolution{ex-JNF-iv}{14.6}
 \begin{itemize}
  \item[(a)] $(t-1)^2(t+1)^2$
  \item[(b)] $3$ and $3$.
  \item[(c)] It follows from~(a) that the JNF of $A$ contains only
   blocks of type $J(1,k)$ or $J(-1,k)$.  Theorem~14.7(b) tells us
   that the number of blocks of type $J(1,k)$ is $4-\rank(A-I)$, which
   is equal to $1$ by~(b).  We thus have only one block of type
   $J(1,k)$, which has to contribute a factor $(t-1)^2$ in the
   characteristic polynomial, so it must be $J(1,2)$.  Similarly, we
   hav only one block of type $J(-1,k)$ and it must be $J(-1,2)$.
   Thus the JNF of $A$ is $J(1,2)\op J(-1,2)$.

  \item[(d)] It follows from~(c) that
   \[ M_A\simeq B(1,2)\op B(-1,2) =
       \C[x]/(x-1)^2\op\C[x]/(x+1)^2\simeq \C[x]/((x-1)^2(x+1)^2).
   \]
   (The last equality uses the Chinese Remainder Theorem, which is
   valid because $(x-1)^2$ and $(x+1)^2$ are coprime.)  It follows
   that $M_A$ is cyclic, as claimed.
 \end{itemize}
\EndDeferredSolution
